\documentclass[border=5pt]{standalone}
\usepackage{forest}
\usetikzlibrary{shapes.geometric}
\begin{document}
\forestset{
  tria/.style={
    node format={
      \noexpand\node [
      draw,
      shape=regular polygon,
      regular polygon sides=3,
      inner sep=0pt,
      outer sep=0pt,
      \forestoption{node options},
      anchor=\forestoption{anchor}
      ]
      (\forestoption{name}) {\foresteoption{content format}};
    },
    child anchor=parent,
  },
  triangle/.style={
    isosceles triangle, isosceles triangle apex angle=45, child anchor=parent, shape border uses incircle, shape border rotate=90, draw,
    delay={
      if={
        >O_> {!u.n children}{1}
      }{
        replace by/.process={ Ow {id} {[, coordinate, tier=##1, append, l'=0pt]}},
        for siblings={tier/.option=id},
        no edge,
      }{},
    },
  },
}%
\begin{forest}
  for tree={l'+=0.5cm, s sep'+=1cm, math content}
  [q_5 [q_8 [q_1 [\overline q_2] [\overline q_6]]
  [q_9 [q_3 [\vdots] [\vdots]] [q_5 [\vdots, triangle] [\vdots]] ] ]
  [q_3 [\overline q_2] [q_4 [\overline q_2] [\overline q_6]] ]
  ]
\end{forest}
\begin{forest}
  for tree={l'+=0.5cm, s sep'+=1cm, math content}
  [q_5 [q_8 [q_1 [\overline q_2] [\overline q_6]]
  [q_9 [q_3 [\vdots] [\vdots]] [q_5 [\vdots, tria] [\vdots]] ] ]
  [q_3 [\overline q_2] [q_4 [\overline q_2] [\overline q_6]] ]
  ]
\end{forest}
\end{document}