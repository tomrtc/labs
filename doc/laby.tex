\documentclass[12pt,a4paper,twocolumn]{book} % 12pt font size, A4 paper and two-sided margins twocolumn,oneside

\usepackage[a4paper,text={16.5cm,25.2cm},centering]{geometry}
\makeatletter				% pour faire de @ une lettre simple (et non un caractère associé à une macro interne
\usepackage{amsmath,amsthm,amsfonts,amssymb,mathtools}
\usepackage{color,graphicx,xcolor}
\usepackage{tikz}
\usetikzlibrary{decorations.pathreplacing}
\usepackage[prefix=rt]{xcolor-material}
\usepackage{blindtext, rotating}
\usepackage{slantsc}
\newcommand\textscsl[1]{\textsc{\slshape#1}}
\usepackage{fontspec,xltxtra,xunicode}
\defaultfontfeatures{}
%\usepackage{palatino} % Use the Palatino font
% to choose a font, you just need its full name on your system
% fc-list  | grep -i texgyr
% Here, the scales of the fonts have been chosento equalise their lowercase letter heights
\setmainfont{TeX Gyre Adventor}[Scale=MatchLowercase]
\usepackage{makeidx}			% pour permettre de construire un index
\usepackage{fancyhdr, graphicx}
\usepackage{fancybox}			% pour faire des boites entourée de différents types de cadres
\usepackage{framed}			% pour mettre des sortes de minipages encadrées sur plusieures pages, sur fond coloré
\usepackage{boxedminipage}		% pour mettre des bordures aux minipages
\usepackage{relsize,fancyvrb}		% pour mettre des textes non interprétés encadrés et changer la taille des caractères à l'intérieur

\usepackage{lettrine}			% pour mettre des lettrines
\usepackage{upgreek}
\usepackage{listings}			% pour mettre du code informatique non interprété
%%\usepackage[shellescape,latex]{gmp}
\usepackage{mpgraphics}
\usepackage{caption}			% pour gérer encore mieux les légendes
\usepackage{showlabels}                 % track of labels\usepackage[final]{showlabels}
\usepackage{tabu}   
\makeatother				% pour refaire de @ une lettre différente des autres (exploitée dans des macros)

 \def\ogla{{\fontfamily{fi4}\selectfont<<}}
  \def\fgla{{\fontfamily{fi4}\selectfont>>}}
  \def\oglb{{\fontfamily{bch}\selectfont<<}}
  \def\fglb{{\fontfamily{ccr}\selectfont>>}}
  \def\oglc{{\fontfamily{pag}\selectfont<<}}
  \def\fglc{{\fontfamily{fvs}\selectfont>>}}
  \def\ogld{{\fontfamily{pnc}\selectfont<<}}
  \def\fgld{{\fontfamily{pzc}\selectfont>>}}


\usepackage{tabstackengine}
\setstackEOL{;}% row separator
\setstackTAB{,}% column separator
\setstacktabbedgap{1ex}% inter-column gap
\setstackgap{L}{1.0\normalbaselineskip}% inter-row baselineskip
\let\nmatrix\parenMatrixstack


\begin{document}



section{\textsc{Au tout début\dots}}



		\begin{itemize}
			\item<+->  via un lecteur de cartes perforées ;
			\item<+->  ou un clavier de machine à écrire électrique.
		\end{itemize}

	

		\begin{itemize}
			\item<+->  sur du papier à bandes perforées ;
			\item<+->  via une imprimante rapide ;
			\item<+->  ou une machine à écrire électrique.
						\item<+->  la police de caractères était l'unique police de l'imprimante ou de la machine à écrire !
		\end{itemize}
{\textsc{On n'avait donc pas la possibilité :}}
		\begin{itemize}
			\item<+-> d'avoir des polices proportionnelles ;
			\item<+-> ni d'avoir du gras, de l'italique ;
			\item<+-> ni des caractères de différentes tailles, indices, exposants\dots
		\end{itemize}
{\textsc{Ensuite sont arrivés les écrans individuels :}}
		\begin{itemize}
			\item<+-> la police de caractères était \textit{bitmap} et unique ;
			\item<+-> en mémoire dans l'écran qui fonctionnait en mode \textit{texte} ;
			\item<+-> on avait 20 à 25 lignes de 40 ou 80 caractères\dots
			\item<+-> Tout ceci codé en \textsc{ascii}, 1963, à la norme étendue flottante\dots
		\end{itemize}
		
		{\textsc{les imprimantes à aiguilles :}}
		\begin{itemize}
			\item<+-> travaillaient en mode \textit{texte} ou \textit{graphique} ;
			\item<+-> avaient 9 puis 24 \textit{aiguilles}, les lignes en fait\dots
		\end{itemize}
	{\textsc{L'apparition de TeX !}
	 {\textsc{Donald \textsc{Knuth}\dots}}
		\begin{itemize}
			\item<+-> crée en 1978 la première version de TeX avec des caractères crées par une ébauche de Metafont ;
			\item<+-> l'idée est déjà d'avoir un système autonome et léger de créations de documents ;
			\item<+-> à l'époque, rien n'existait\dots
		\end{itemize}	
	{\textsc{MetafontLes bases}
	{\textsc{Le principe est celui du roseau fendu}}
		\begin{itemize}
			\item<+-> ou de la plume Sergent Major ;
			\item<+-> qui permet de créer des pleins et des déliés\dots
		\end{itemize}	
		
	{\textsc{Comme en calligraphie}}
		\begin{itemize}
			\item<+-> la plume suit une courbe orientée ;
			\item<+-> en étant plus ou moins inclinée ;
			\item<+-> cette inclinaison est variable au besoin\dots
			\item<+-> On utilise des courbes de Bézier, sur lesquelles on reviendra\dots
		\end{itemize}	

{\textsc{Une telle police :}}
		\begin{itemize}
			\item<+-> est ensuite convertie en bitmap haute précision pour impression,
				la puissance des machines à ce moment interdisait de faire cela à la volée ;
			\item<+-> on lui adjoint des instructions de crénage, rapprocher ou éloigner deux caractères
				qui s'emboîtent plus ou moins bien ;
			\item<+-> et des instructions de ligature, réunir deux caractères qui se suivent en un seul\dots
			\item<+-> est limitée à 256 caractères !
			\item<+-> Bien qu'au final, dans le document, c'est du bitmap, les polices Metafont sont les premières polices vectorielles !
			\end{itemize}	
			
		{Un exemple de police non optique}
	\begin{verbatim}
	{\fontfamily{jkp}\selectfont
	\newcommand{\PO}{Police non optique}
	\scalebox{4}{\fontsize{6}{6}\selectfont\PO}\\[1ex]
	\scalebox{2}{\fontsize{12}{12}\selectfont\PO}\\[1ex]
	\scalebox{.75}{\fontsize{32}{32}\selectfont\PO}}
	\end{verbatim}
	{\fontfamily{jkp}\selectfont
	\newcommand{\PO}{Police non optique}
		\scalebox{4}{\fontsize{6}{6}\selectfont\PO}\\[1ex]
		\scalebox{2}{\fontsize{12}{12}\selectfont\PO}\\[1ex]
		\scalebox{.75}{\fontsize{32}{32}\selectfont\PO}}	
	{Un exemple de polices optiques}
	\begin{verbatim}
	{\fontfamily{cmr}\selectfont
	\newcommand{\PO}{Police optique}
	\scalebox{4}{\fontsize{6}{6}\selectfont\PO}\\[1ex]
	\scalebox{2}{\fontsize{12}{12}\selectfont\PO}\\[1ex]
	\scalebox{.75}{\fontsize{32}{32}\selectfont\PO}}
	\end{verbatim}
	{\fontfamily{cmr}\selectfont
		\newcommand{\PO}{Police optique}
		\scalebox{4}{\fontsize{6}{6}\selectfont\PO}\\[1ex]
		\scalebox{2}{\fontsize{12}{12}\selectfont\PO}\\[1ex]
		\scalebox{.75}{\fontsize{32}{32}\selectfont\PO}}		
	{\textsc{Une telle police :}}
		\begin{itemize}
			\item<+-> en 12 pt : {\fontsize{12}{14}\fontfamily{jkp}\selectfont AaBbCc} ;
			\item<+-> en 25 pt : {\fontsize{25}{28}\fontfamily{jkp}\selectfont AaBbCc} ;
			\item<+-> le caractère agrandi parait plus gras\dots
			\item<+-> Metafont permet facilement de créer des polices optiques en modifiant le crayon !
			\item<+-> Mais aujourd'hui, les polices optiques ont presque complètement disparues, on en reparlera\dots		
			\end{itemize}	
			{\textsc{Postcript}}
		\begin{itemize}
			\item<+-> est un langage de chez Adobe apparu en 1982 ;
			\item<+-> puis en 1984 pour les polices.
		\end{itemize}
{\textsc{Un caractère Postcript de type 1}}
		\begin{itemize}
			\item<+-> est formé de contours fermés orientés ;
			\item<+-> assemblage de courbes de Bézier cubiques ;
			\item<+-> et on noircit ce qui est à droite de ces contours\dots
		\end{itemize}
	\textsc{Un exemple :}}
  	\begin{center}\begin{tikzpicture}
  		\filldraw [pink] (1,2) circle (2pt)
  		                 (0,1) circle (2pt);
  		\filldraw [green](0,0) circle (2pt)
  		                 (5,2) circle (2pt);
  		\draw [->] (0,0) .. controls (0,1) and (1,2) .. (5,2) ;
  		\draw	[gray,->] (0,0) -- (0,1) ;
  		\draw	[gray,->] (5,2) -- (1,2) ;
  	\end{tikzpicture}\end{center}	

	On a symbolisé les extrémités en vert et les deux points de contrôle en rose
	
	On assemble des courbes de ce type pour obtenir un ou des chemins fermés.
	
	On remplit alors \textsl{à droite} des chemins fermés.				Une Bezier cubique se décompose en deux Beziers cubiques !
	
	 	\begin{center}\begin{tikzpicture}
  		\draw [->] (.59,3.36) .. controls (.88,4.11) and (1.50,4.58) .. (2.40,4.58) ;
  		\draw [red] (.59,3.36) -- (.88,4.11) -- (1.50,4.58) -- (2.40,4.58)
  					-- (2.71,4.58) -- (3.78,3.97) -- (3.78,2.98) --(3.78,.94) -- (3.78,.42) -- (4.05,.32) 
  					-- (4.47,.43) -- (4.51,.17) -- (4.51,.17) -- (3.72,.04) -- (3.25,-.07) 
  					-- (2.97,-.07) -- (3.00,.68) -- (3.00,.68) -- (2.13,-.12) -- (1.58,-.12) --
  					(.87,-.12) -- (.35,.33) -- (.35,1.07) -- (.35,1.88) -- (.97,2.12) -- (1.70,2.37)
  					-- (3.00,2.81) -- (3.00,3.58) -- (2.78,4.13) -- (1.91,4.10) -- (1.46,4.07) -- 
  					(1.17,3.60) -- (.94,3.16) ;
  		\draw [->] (2.40,4.58) .. controls (2.71,4.58) and (3.78,3.97) .. (3.78,2.98) ;
  		\draw [->] (3.78,2.98) -- (3.78,.94) ;
  		\draw [->] (3.78,.94) .. controls (3.78,.42) and (4.05,.32) .. (4.47,.43) ;
  		\draw [->] (4.47,.43) -- (4.51,.17) ;
  		\draw [->] (4.51,.17) .. controls (4.51,.17) and (3.72,.04) .. (3.25,-.07) ;
  		\draw [->] (3.25,-.07) .. controls (2.97,-.07) and (3.00,.68) .. (3.00,.68) ;
  		\draw [->] (3.00,.68) .. controls (2.13,-.12) .. (1.58,-.12) ;
  		\draw [->] (1.58,-.12) .. controls (.87,-.12) and (.35,.33) .. (.35,1.07) ;
  		\draw [->] (.35,1.07) .. controls (.35,1.88) and (.97,2.12) .. (1.70,2.37) ;
  		\draw [->] (1.70,2.37) -- (3.00,2.81) ;
  		\draw [->] (3.00,2.81) .. controls (3.00,3.58) and (2.78,4.13) .. (1.91,4.10) ;
  		\draw [->] (1.91,4.10) .. controls (1.46,4.07) and (1.17,3.60) .. (.94,3.16) ;
  		\draw [->] (.94,3.16) -- (.59,3.36) ;
  		\draw [red] (3.00,1.03) -- (2.77,.76) -- (2.28,.44) -- (1.91,.44) -- (1.48,.44) -- (1.20,.81) --
  		      (1.20,1.23) -- (1.20,1.95) -- (3.00,2.49) -- (3.00,2.49) ;
  		\draw [<-] (3.00,2.49) -- (3.00,1.03) ;
  		\draw [<-] (3.00,1.03) .. controls (2.77,.76) and (2.28,.44) .. (1.91,.44) ;
  		\draw [<-] (1.91,.44) .. controls (1.48,.44) and (1.20,.81) .. (1.20,1.23) ;
  		\draw [<-] (1.20,1.23) .. controls (1.20,1.95) and (3.00,2.49) .. (3.00,2.49) ;
  	\end{tikzpicture}\end{center}
	
		Notez les sens de parcours !
	\begin{itemize}
			\item<+-> la notion de police définie par des contours, ce qui rend les polices optiques difficiles à écrire ;
			\item<+-> tous les caractères on un nom, \textit{agrave} pour \textit{à}, ce qui permet de dépasser la limite des 256 caractères,
				ceci restera vrai dans les polices suivantes\dots
			\item<+-> Cependant, il ne peut, par exemple, exister qu'un seul \textit{a} utilisable très facilement !
			\item<+-> Le fichier de dessin des caractères doit s'accompagner d'un fichier de métriques, ligatures et crénages.
		\end{itemize}
		
\section{TrueType -- .ttf}		
	{\textsc{Les polices TrueType}}
		\begin{itemize}
			\item<+-> sont une création d'Apple, apparues à la fin des années 80 ;
			\item<+-> dans le but de concurrencer les polices Postscript.
		\end{itemize}	
	{\textsc{Un caractère TrueType}}
		\begin{itemize}
			\item<+-> est formé de contours fermés orientés ;
			\item<+-> assemblage de courbes de Bézier quadratiques ;
			\item<+-> et on noircit ce qui est à droite de ces contours\dots
		\end{itemize}	
{\textsc{Grande nouveauté}}
		\begin{itemize}
			\item<+-> les métriques sont intégrées au fichier .ttf ;
			\item<+-> un seul fichier suffit donc !
		\end{itemize}	
{\textsc{La guerre des polices n'aura pas lieu}}
		\begin{itemize}
			\item<+-> Adobe et Apple se mettront d'accord pour développer les polices TrueType et suivantes ensemble !
		\end{itemize}		


Utilisation dans pdf\LaTeX		
{\textsc{Possible assez facilement}}
		\begin{itemize}
			\item<+-> mais il faut fabriquer les fichiers nécessaires ;
			\item<+-> Font Definition -- .fd
			\item<+-> TeX Font Metrics -- .tfm, voire peut-être -- .vf
			\item<+-> le fichier .map et au besoin le ou les fichiers d'encodage -- .enc !
		\end{itemize}
{\textsc{Les polices OpenType}}
		\begin{itemize}
			\item<+-> ont été crées en 1996 ;
			\item<+-> sont développées conjointement par Adobe et Apple.
		\end{itemize}
		
	{\textsc{Un caractère OpenType}}
		\begin{itemize}
			\item<+-> est formé de contours fermés orientés ;
			\item<+-> assemblage de courbes de Bézier cubiques, ou quadratiques parfois ;
			\item<+-> et on noircit ce qui est à droite de ces contours\dots
			\item<+-> Rien de nouveau pour le moment !
		\end{itemize}	
		
		
		
		
{\textsc{Les polices OpenType}}
		\begin{itemize}
			\item<+-> sont dites \textit{intelligentes} ;
			\item<+-> on leur passe des commandes pour fournir à la même demande des résultats différents !
			\item<+-> Par exemple, on a souvent 4 caractère \ogla 1 \fgla{} différents :
			\begin{itemize}
				\item<+-> le {\fontfamily{jkp}\selectfont 1}, en largeur fixe ou proportionnelle ;
				\item<+-> le {\fontfamily{jkp}\selectfont \oldstylenums{1}} elzévirien, en largeur fixe ou proportionnelle.
			\end{itemize}
			\item<+-> Par défaut, c'est le premier, mais on aura des commandes pour passer en proportionnel et en elzévirien !
		\end{itemize}
		
				
		
		
		{\textsc{Les commandes}}
			\begin{itemize}
				\item<+->  Elles correspondent des scripts exécutables gérant :
				\begin{itemize}
					\item<+-> les corps optiques ;
					\item<+-> le crénage ;
					\item<+-> les substitutions ;
					\item<+-> mais aussi les substitutions contextuelles !
				\end{itemize}
				\item<+->  Ce qui permet d'avoir par exemple sans modification du source :
				
			
				{\fontfamily{jkpvos}\selectfont\itshape
				Quelles= superbes= questions= !}
				
	
%%{\fontfamily{jkp}\selectfont\scslshape Quelles superbes questions !}
			
				
				Regardez attentivement petites capitales penchées et les \ogla s \fgla !
				
				Il faut, bien sûr, que la fonte considérée le permette !
				\item<+->  En principe, les polices OpenType \textit{PRO} contiennent des variantes optiques !
\end{itemize}
{\textsc{Commandes courantes}}
			
			\begin{itemize}
				\item<+-> Les commandes de ces polices sont des mots de 4 lettres, certaines, souvent toutes, sont prédéfinies :
				\begin{itemize}
					\item<+-> \texttt{onum} et \texttt{pnum} pour les nombres elzéviriens et proportionnels ;
					\item<+-> \texttt{smcp} pour les petites capitales\dots
				\end{itemize}
			\end{itemize}		
		
		
	{Trouver les spécifications d'une fontes OpenType}	
{\textsc{Cela dépend du système !}}
  \begin{itemize}
  \item Sous Linux, installer \textit{Fontforge} par exemple :
			\begin{itemize}
				\item<+-> Suivre \textit{View}, \textit{Display Substitutions}\dots
			\end{itemize}
	\item Sous Mac OS ou Windows, installer la version de démonstration de \textit{FontLab Studio} :
			\begin{itemize}
				\item<+-> l'info est sur le panneau \textit{OpenType}, qu'on peut rendre visible au besoin,
				\item<+-> en suivant \textit{Window}, \textit{Panels}, \textit{OpenType}\dots
			\end{itemize}
			Sous Mac, c'est aussi dans les propriétés de la police.
	\end{itemize}		
		
		
		{\textsc{Liste des commandes standard et leur sens !}}
			
			\begin{itemize}
				\item<+-> On trouve la liste des commandes possibles sur \verb=http://en.wikipedia.org/wiki/=
				
				\qquad\qquad\qquad\verb=List_of_typographic_features=
				\item<+-> ou encore \begin{verbatim}
				http://www.adobe.com/devnet/opentype/afdko/
				           topic_feature_file_syntax.html
				\end{verbatim}
			\end{itemize}
		
		
	section{OpenType et \LuaLaTeX}	
		
		
	{\textsc{Des nombres :}}
			\begin{verbatim}
\fontspec{TeX Gyre Pagella}00011123456789

\fontspec[Numbers={Proportional}]
                  {TeX Gyre Pagella}00011123456789

\fontspec[Numbers={OldStyle}]
                  {TeX Gyre Pagella}00011123456789
			\end{verbatim}	
		
		
		\fontspec{TeX Gyre Pagella}00011123456789
		
		\fontspec[Numbers={Proportional}]
                  {TeX Gyre Pagella}00011123456789

\fontspec[Numbers={OldStyle}]
                  {TeX Gyre Pagella}00011123456789			
                  
                  
                  
                  
                  
 {\textsc{À noter :}}
\begin{itemize}
\item<+->	\begin{verbatim}
			\usepackage[utf8]{inputenc} % ou latin1...
			\usepackage[T1]{fontenc}     % ou OT1...
			\end{verbatim}
			disparaissent...
\item<+->	\begin{verbatim}
			\usepackage{fontspec}
			\end{verbatim}
			est obligatoire !
\item<+->	Si on veut utiliser Beamer :
			\begin{verbatim}
			\usepackage{luatextra}
			\end{verbatim}
			est obligatoire !
\item<+->	On peut continuer à utiliser \texttt{babel} ;
\item<+->	Réglez au départ votre éditeur pour utiliser l'encodage \texttt{utf8} par défaut, c'est obligatoire !
\end{itemize}                 
                  
                  
    {\textsc{Pas besoin de nouveau package !}}
  \begin{itemize}
  \item Pour un document ou une partie de document, on utilise les commandes :
			\begin{itemize}
				\item<+-> \verb=\setmainfont[spécifications]{Nom de Police}=
				\item<+-> \verb=\setsansfont[spécifications]{Nom de Police}=
				\item<+-> \verb=\setmonofont[spécifications]{Nom de Police}=
				\item<+-> \verb=\fontspec[spécifications]{Nom de Police}=
				\item<+-> \verb=\newfontfamily\mapolice=
				
				\hfill\verb=[spécifications]{Nom de Police}=
				\item<+-> \verb=\addfontfeature{specifications}=
				\item<+-> \dots
			\end{itemize}
	\item Pas de package, mais les polices ont, ou devraient avoir, une doc qui précise les spécifications admises...
			et qu'il faut lire !
	\end{itemize}              
                  
                  
    {\textsc{Mathématiques en OpenType}}
			\begin{itemize}
				\item<+->  On a déjà vu que le texte en OpenType et les mathématiques habituelles de \LaTeX cohabitent sans problème !
				\item<+->  Le package \texttt{unicode-math} permet de composer avec les polices mathématiques OpenType existantes :
				\begin{itemize}
					\item Cambria Math (avec \textit{Microsoft Office})
					\item Minion Math (police commerciale \texttt{typoma})
					\begin{itemize}
					\item ou bien le package \texttt{MnSymbol}, mais\dots
					\end{itemize}
					\item Latin Modern Math
					\item TeX Gyre Pagella Math, et autres TeX Gyre : Bonum, Schola,Termes
					\item Asana Math
					\item Neo Euler
					\item STIX
					\item XITS\dots
				\end{itemize}
			\end{itemize}              
                  
   {\textsc{Un préambule minimum :}}
			\begin{verbatim}
			\documentclass[12pt]{article}
			\usepackage[frenchb]{babel}
			\usepackage{unicode-math}
			\setmainfont{TeX Gyre Termes}
			\setmathfont{TeX Gyre Termes Math}
			\begin{document}
			...
			\end{document}
			\end{verbatim}               
                  
                  
 {\textsc{D'autre part :}}
		\begin{itemize}
			\item<+->  les polices historiques de \LaTeXe{} sont les polices \textit{metafont}, fichiers \texttt{mf,pk} ;
			\item<+->  les autres polices vectorielles sont:
			\begin{itemize}
				\item<+-> les polices \textit{postscript}, en \texttt{.pfb} le plus souvent ;
				\item<+-> les polices \textit{true type}, en \texttt{.ttf} ;
				\item<+-> les polices \textit{open type}, en \texttt{.otf} ;
			\end{itemize}
			\item<+->  \texttt{dvips} gère les polices \textit{postscipt};
			\item<+->  \texttt{pdfTeX} gère aussi les polices \textit{true type} et \textit{open type};
			\item<+->  \texttt{XeTeX} gère aussi \textit{directement} les polices \texttt{ttf} et surtout \texttt{otf} ;
			\item<+->  comme \texttt{LuaTeX} devrait le faire bientôt.
		\end{itemize}                 
           {\textsc{Et enfin :}}
		\begin{itemize}
			\item<+->  \textit{fontinst} permet d'installer (assez) facilement des polices Postscript ;
			\item<+->  \textit{fontools} permet d'installer (assez) facilement des polices \texttt{ttf} et surtout \texttt{otf} ;
		\end{itemize}   
		
	\section{\textsc{Les familles}}

\subsection{Texte}	
		
		
	{\textsc{Les 3 familles (prédéfinies) de polices}   \LaTeX{} \textsc{sont}}
			\begin{itemize}
				\item<+->  \textrm{Les polices romaines : rm} ;
				\item<+->  Les polices sans-serif : sf ;
				\item<+->  \texttt{Les polices machine à écrire : tt}.
				\end{itemize}	
		
	\textsc{Comment modifier une famille par défaut ?}}
	
		
	{\textsc{En utilisant un Package}}
			\begin{itemize}
				\item<+-> \textit{bookman, newcent, times, palatino} modifient les 3 familles de polices,
				
								 avec la police titre en romaine ;
				\item<+-> \textit{helvet, avant} modifient les familles sans-serif ;
				\item<+-> \textit{courier, luximono} modifient les familles télétype ;
				\item<+-> \dots  
				\item<+-> \textit{helvet, luximono} prennent en option un coefficient de taille pour s'adapter à toutes	
								les polices romaines.
			\end{itemize}	
		
	{\textsc{Sans utiliser de Package}}
  \begin{itemize}
  \item<+-> Pour un document ou une partie de document, on utilise les commandes :
			\begin{itemize}
				\item<+-> \verb=\renewcommand{\rmdefault}{=\textsl{jkp}\texttt{\}}
				\item<+-> \verb=\renewcommand{\sfdefault}{=\textsl{jkpss}\texttt{\}}  
				\item<+-> \verb=\renewcommand{\ttdefault}{=\textsl{jkptt}\texttt{\}}
				\item<+-> \verb=\renewcommand{\familydefault}{=\textsl{jkpss}\texttt{\}}
			\end{itemize}
	\item<+-> on décrira plus loin comment avoir une modification \ogla plus locale \fgla.
	\end{itemize}	
		
	\subsection{Math}	
		
	{\textsc{On a toujours au moins les familles :}}
			\begin{itemize}
				\item<+-> \textit{operators}, pour $0123\;+-=\;\Gamma\Delta\dots$
				\item<+-> \textit{letters} pour $abc\;\alpha\beta\gamma\dots$
				\item<+-> \textit{symbols} pour les symboles de base, comme $\to\;\mapsto\;\Rightarrow\;\exists\dots$
				\item<+-> \textit{largesymbols} pour les symboles mathématiques de taille variable, (et de base) 
									$\sum\;\displaystyle\sum$ $\int\;\displaystyle\int\dots$
			\end{itemize}	
	\textsc{Comment modifier une famille par défaut ?}
	
	{\textsc{En utilisant un Package}}
			\begin{itemize}
				\item<+-> \textit{txfonts} utilise \textit{times}, en texte et en math \ogla letters \fgla ;
				\item<+-> \textit{pxfonts} utilise \textit{palatino} en\dots ;
				\item<+-> \textit{fourier} utilise \textit{utopia} ;
				\item<+-> \textit{euler} modifie les polices mathématiques, souvent utilisé avec \textit{concrete} en texte ;
				\item<+-> \textit{kpfonts} utilise ses polices originales et modifie tout ;
				\item<+-> voir aussi \textit{mathdesign} qui prend en option une des polices romaines : 
								\textit{utopia, garamond, charter}\dots
			\end{itemize}
	\textsc{Comment modifier une police mathématiques ?}
	
	{\textsc{A partir de polices de texte}}
			\begin{itemize}
				\item<+-> Vous utilisez \textit{fourier}, par exemple, 
				et vous voulez avoir \textit{courier} comme police \texttt{télétype}, 
				tant en mode texte qu'en mode mathématique :
				\begin{itemize}
					\item<+-> \verb=\renewcommand{\ttdefault}{pcr}=
            \begin{itemize}
            	\item pour le mode texte ;
            \end{itemize}
					\item<+-> \verb=\DeclareMathAlphabet{\mathtt}{T1}{pcr}{m}{n}=
					\item<+-> \verb=\SetMathAlphabet{\mathtt}{bold}{T1}{pcr}{b}{n}=
            \begin{itemize}
            	\item pour le mode mathématique ;
            \end{itemize}
				\end{itemize}
				\item<+-> \`A placer dans le préambule, bien sûr\dots
			\end{itemize}
	
	Vous pouvez aussi de la même façon ajouter un \verb=\mathsc=, par exemple
	\textsc{Comment ajouter une police mathématique ?}
	{\textsc{A partir de polices mathématiques}}
			\begin{itemize}
				\item<+-> Vous voulez ajouter la police \verb=\mathscr= de \textit{kpfonts}, vous tapez :
				\item<+-> \verb=\DeclareMathAlphabet{\mathscr}{U}=
									
									\hfill\verb={jkpsyd}{m}{n}=
				\item<+-> \verb=\SetMathAlphabet{\mathscr}{bold}{U}=
									
									\hfill\verb={jkpsyd}{b}{n}=
				\item<+-> On a déclaré une nouvelle commande \verb=\mathscr= , utilisable aussi en gras\dots
			\end{itemize}
		Les informations nécessaires sont prises dans les fichiers \textit{sty} des packages utilisés.
		
		
		\textsc{Comment ajouter une police de symboles ?}
		
		
		{\textsc{Quand utiliser ceci ?}}
			\begin{itemize}
				\item<+-> Vous voulez avoir le \verb=\int= droit de \textit{kpfonts} :
				\begin{itemize}
					\item<+-> \verb=\DeclareSymbolFont{kpint}{OMX}{jkp}{m}{n}=
					\item<+-> \verb=\SetSymbolFont{kpint}{bold}{OMX}{jkp}{bx}{n}=
					\item<+-> \verb=\let\intop\undefined=
					\item<+-> \verb=\DeclareMathSymbol{\intop}{\mathop}{kpint}{82}=
					\item<+-> \verb=\def\int{\intop\nolimits}=
				\end{itemize}
				\item<+-> \`A placer dans le préambule, bien sûr\dots
				\item<+-> Le nom \textsl{kpint} est arbitraire ;
				\item<+-> Il faudrait le refaire pour les intégrales doubles etc
			\end{itemize}
		
		
	\section{\textsc{Les attributs}}
{\textsc{Les 5 attributs d'une police   \LaTeX{}}}
			\begin{itemize}
				\item<+->  Le codage : ot1, t1, ts1, oml, oms, omx\dots 
				\item<+->  La famille : cmr, cmss, ptm, ppl, jkp, jkpss\dots
				\item<+->  La graisse : m, b, bx\dots
				\item<+->  La forme : n, it, sc, sl\dots
					\begin{itemize}
						\item<+->  Ce qui empèche d'avoir des petites capitales italiques !
						\item<+->  \textscsl{mais KPFONTS en a quand même (forme scsl)}\dots
					\end{itemize}
				\item<+->  La taille en points.
			\end{itemize}
\subsection{Texte}
	
		
		
		{\textsc{Modifier les attributs d'une police de texte}}
  \begin{itemize}
		\item<+-> Les instructions sont les suivantes :
			\begin{itemize}
				\item<+->  l'encodage : \verb=\fontencoding{...}=
				\item<+->  la famille : \verb=\fontfamily{...}=
				\item<+->  la graisse : \verb=\fontseries{...}=
				\item<+->  la forme : \verb=\fontshape{...}=
				\item<+->  la taille : \verb=\fontsize{=\textit{taille}\verb=}{=\textit{saut de ligne}\verb=}=
			\end{itemize}
		\item<+-> toujours suivis de \verb=\selectfont=\dots
		\item<+-> Ceci s'ajoute aux commandes de haut niveau que vous connaissez !
	\end{itemize}
	
	
	{\textsc{Mathématiques : passer en gras}}
  \begin{itemize}
		\item<+-> En mode texte :
			\begin{itemize}
				\item<+->  tapez : \verb=\mathversion{bold}=
				\item<+->  et tapez : \verb=\mathversion{normal}= 
				
							pour quitter le mode gras.
			\end{itemize}
		\item<+-> Le package \textit{bm} permet de passer en gras dans une partie de formule ;
		\item<+-> il y a aussi la police \verb=\mathbf=
	\end{itemize}
	
	
	
	
	\subsection{Fichier \texttt{.fd}}
	
	
	
	{\textsc{Trouver le bon fichier} \texttt{.fd}}
			\begin{itemize}
				\item<+->  \`A partir du codage, par exemple \texttt{t1} ; 
				\item<+->  et de la famille, par exemple \texttt{jkp} ;
				\item<+->    \LaTeX{} cherche le fichier \texttt{t1jkp.fd}.
			\end{itemize}
	
	
{\textsc{Le fichier} \texttt{.fd}}
\begin{itemize}
	\item<+->  Le fichier \texttt{.fd} contient :
			\begin{itemize}
				\item<+->  une ligne de déclaration de famille :
				
							\verb=\DeclareFontFamily{T1}{jkp}{}=
				\item<+->  des lignes de déclarations de forme :
				
							\verb=\DeclareFontshape{T1}{jkp}{m}{n}{<->jkpmn8t}{}=
							
				\item<+->  des lignes de substitution :
				
							\verb=\DeclareFontshape{T1}{jkp}{m}{it}=
							
							\hfill\verb+{<->ssub * jkp/m/sl}{}+
			\end{itemize}
	\item<+->  remarquons qu'on a ici \texttt{jkpmn8t}, le nom interne d'une \textit{police}   \LaTeX{}.
	\item<+->  On a des choses équivalentes en math avec les \verb=\Declare=\dots{} déjà vus.
	
	Les packages de polices mathématiques ont aussi leurs fichiers \texttt{.fd}.
\end{itemize}	
	
	
	
	
	{\textsc{Polices réelles ou virtuelles}}
\begin{itemize}
	\item<+-> Les noms internes de police   \LaTeX{} correspondent à des polices rélles ou virtuelles :
			\begin{itemize}
				\item<+->  une police réelle correspond à un fichier \texttt{.tfm} ;
				\item<+->  une police virtuelle correspond à un fichier \texttt{.tfm} et un fichier \texttt{.vf} ;
					\begin{itemize}
						\item<+->  \texttt{.tfm} = TeX Font Metric ;
						\item<+->  \texttt{.vf} = Virtual Font.
					\end{itemize}
				\item<+->  Leurs formats lisibles sont les fichiers \texttt{.pl} et \texttt{.vpl} :
					\begin{itemize}
						\item<+->  un fichier \texttt{.pl} correspond au fichier \texttt{.tfm};
						\item<+->  un fichier \texttt{.vpl} correspond aux 2 fichiers \texttt{.tfm} et \texttt{.vf}.
					\end{itemize}
			\end{itemize}
	\item<+-> Pour l'instant, aucun de ces fichiers ne contient de dessin de caractère !
\end{itemize}
	
	
	
	
	
\subsection{Fichiers \texttt{.pl .vpl}}
	
	
	
	
	
	
	
Un fichier \texttt{.pl} ou \texttt{.vpl} contient des lignes du type :

\texttt{(FONTDIMEN ...)}
{\fontfamily{jkptt}\selectfont 

{\color[rgb]{0,.5,0}(MAPFONT D 0 (FONTNAME ...))

(MAPFONT D 1 (FONTNAME ...))}

   (LIGTABLE

		...
		
   (LABEL O 55)(LIG O 55 O 173)(STOP)
   
   ...
   
   (LABEL C A)(KRN C V R -0.12)(STOP)
   
   ...)}
   
{\fontfamily{jkptt}\selectfont ...

(CHARACTER O 20

   (CHARWD R 0.314)(CHARHT R 0.4625)
   
   (CHARDP R 0.00475)(CHARIC R 0.053)
   
   {\color[rgb]{0,.5,0}(MAP (SELECTFONT D 1) (SETCHAR O 20)))}
   
...}
	
	
	
	{\textsc{Les lignes d'un fichier} \texttt{.pl,.vpl} \textsc{en mode math}}
\begin{itemize}
	\item<+-> Dans une police mathématique, on peut aussi trouver :
\begin{itemize}
	\item<+-> \texttt{NEXTLARGER}, pour les symboles à dimension variable ;
	\item<+-> et, pour les symboles à extention infinie, 
	
	\texttt{VARCHAR}, puis :
	\begin{itemize}
	\item<+-> \texttt{REP}, répétition ;
	\item<+-> \texttt{TOP}, haut ;
	\item<+-> \texttt{BOT}, bas ;
	\item<+-> \texttt{MID}, milieu.
	\end{itemize}
	\item<+-> Les accolades, par exemple, nécessitent l'ensemble de ces éléments.
\end{itemize}
\end{itemize}
	
	Les polices référencées dans le fichier \texttt{.vpl} sont aussi réelles ou virtuelles
Mais on finit par arriver sur une police réelle !

  \LaTeX{} cherche alors la police dans un fichier \texttt{.map}
	
	{\textsc{Les lignes d'un fichier} \texttt{.map} \textsc{contiennent}}
			\begin{itemize}
				\item<+->  \texttt{jkpmsl8r} : le nom   \LaTeX{} de la police (fichier tfm) ;
				\item<+->  \texttt{Kp-Regular} : le nom postscript de la police ;
				\item<+->  \texttt{<8r.enc} : le réencodage
				 ;
				\item<+->  \texttt{<jkpmn8a.pfb} : le nom du fichier des dessins des caractères ;
				\item<+->  \texttt{" TeXBase1Encoding ReEncodeFont}
				
									\hfill\texttt{0.167 SlantFont "} :
				
									encodage et transformations pour \textit{dvips} ou \textit{pdfTeX}.
			\end{itemize}
	
	
	\subsection{Fichier \texttt{.enc}}
	
	{\textsc{Un fichier} \texttt{.enc}}
			\begin{itemize}
				\item<+->  fait le lien :
				\begin{itemize}
					\item<+->  entre les slots (positions, de 0 à 255) utilisés par \TeX ;
					\item<+->  et les noms \textit{postscript} des caractères dans le \texttt{.pfb};
				\end{itemize}
				\item<+->  et permet :
				\begin{itemize}
					\item<+->  d'utiliser des polices \texttt{.pfb} de plus de 256 caractères ;
					\item<+->  la recherche de mots dans les fichiers \texttt{.ps} ou \texttt{.pdf},
					           le nom des caractères y est inséré ;
				\end{itemize}
				\item<+->  et c'est souvent absent dans les polices de symboles\dots
			\end{itemize}
	
	\section{\textsc{Métriques}}

\subsection{Fontdim}

	
	{\textsc{Les dimensions d'une police de texte sont :}}
	\begin{enumerate}
		\item<+->  \texttt{SLANT}, son inclinaison ;
		\item<+->  \texttt{SPACE}, la largeur de l'espace ;
		\item<+->  \texttt{STRETCH}, la dilatation possible de l'espace ;
		\item<+->  \texttt{SHRINK}, sa compression possible ;
		\item<+->  \texttt{XHEIGHT}, la hauteur du \ogla x\fgla ;
		\item<+->  \texttt{QUAD}, la largeur du \texttt{quad}, aussi égale à \texttt{1 em} ;
		\item<+->  \texttt{EXTRASPACE}, l'espace supplémentaire en début de phrase.
	\end{enumerate}
		Les polices mathématiques \textsc{symbols} et \textsc{largesymbols} contiennent respectivement 22 et 13 dimensions,
	
	qui servent à construire les formules mathématiques. 
	
	On peut voir un exemple de placement d'indice plus loin.
	
	\subsection{Ligatures}
\textsc{Ligatures et Crénages}
	
	{\textsc{Ligatures}}
  \begin{itemize}
  \item<+-> Une \textit{ligature} est le regroupement de deux caractères consécutifs en un troisième.
  \item<+-> Une ligature peut être :
		\begin{itemize}
			\item<+-> technique comme {-}- qui donne -- ;
			\item<+-> classique comme {f}i qui donne fi ;
			\item<+-> ou désuette comme {\fontfamily{jkpos}\selectfont {c}t} qui donne {\fontfamily{jkpos}\selectfont ct}.
		\end{itemize}
  \item<+-> En général, on n'utilise pas de ligature en mode math.
	\end{itemize}
	
\subsection{Crénages}	
	
	
	{\textsc{Crénages}}
  \begin{itemize}
  \item<+-> Un \textit{crénage} est la façon dont deux caractères se rapprochent ou s'éloignent selon leur dessin :
		\begin{itemize}
			\item<+-> On a \ogla Tout \fgla{} et non \ogla {T}out \fgla{} ;
			\item<+-> Et on a \ogla dîme \fgla{} et non \ogla {d}îme \fgla.
		\end{itemize}
  \item<+-> En général, on n'utilise pas de crénage en mode math.
	\end{itemize}
	
	\subsection{Caractères}
	
	
{\textsc{Métriques des caractères}}
	 \begin{itemize}
		 \item<+->  La métrique d'un caractère a 4 éléments :
		\begin{itemize}
			\item<+-> \texttt{CHARWD}, sa largeur ;
			\item<+-> \texttt{CHARHT}, sa hauteur au dessus de la ligne de base ;
			\item<+-> \texttt{CHARDP}, sa profondeur en dessous de la ligne de base ;
			\item<+-> \texttt{CHARIC}, sa correction italique.
		\end{itemize}
		 \item<+->  Le sens de la largeur et de la correction italique n'est pas le même en mode texte ou en mode math.
	 \end{itemize}	
	
	
{\textsc{Métriques des caractères en texte}}
	 \begin{itemize}
		 \item<+->  Pour un caractère en mode texte :
		\begin{itemize}
			\item<+-> \texttt{CHARWD}, sa largeur est sa largeur ;
			\item<+-> \texttt{CHARIC}, sa correction italique est appliquée quand on quitte l'italique.
		\end{itemize}
		 \item<+->  On va voir cela dans 40 secondes\dots
	 \end{itemize}	
	
	{\textsc{Métriques des caractères en mode math}}
	 \begin{itemize}
		 \item<+->  En mode math, pour un caractère :
		\begin{itemize}
			\item<+-> \texttt{CHARWD}, sa largeur sert à placer l'indice ;
			\item<+-> \texttt{CHARWD+CHARIC}, sert à placer l'exposant, et,
			
			est la largeur effective du caractère, 
			
			sauf en cas de présence d'exposant ou d'indice assez large.
		\end{itemize}
		 \item<+->  On va voir cela dans 10 secondes\dots
	 \end{itemize}
	
	
	{\textsc{Métriques des caractères en mode texte et math}}
  	\begin{center}\begin{tikzpicture}
  		\filldraw [green] (1.5,0.2) -- (1.5,3.5) -- (0.2,3.5) -- (0.2,3.8) -- (3.1,3.8) -- (3.1,3.5) -- (1.8,3.5) -- (1.8,0.2) -- cycle ;
  		\filldraw [blue] (6,0.2) -- (6,3.5) -- (4.7,3.5) -- (4.7,3.8) -- (7.6,3.8) -- (7.6,3.5) -- (6.3,3.5) -- (6.3,0.2) -- cycle ;
  		\draw	[red] (0,0.2) -- (3.3,0.2) -- (3.3,3.8) -- (0,3.8) -- cycle ;
  		\draw	[red] (4.3,0.2) -- (8,0.2) -- (8,3.8) -- (4.3,3.8) -- cycle ;
  		\draw	[red] (6.8,0.2) -- (6.8,3.8) ;
  		\draw	[gray,<->] (0,1) -- (3.3,1) ; \draw [gray] (1.55,1.2) node {CHARWD};
  		\draw	[gray,<->] (4.3,1) -- (6.8,1) ; \draw [gray] (5.55,1.2) node {CHARWD};
  		\draw	[gray,<->] (6.8,2) -- (8,2) ; \draw [gray] (7.4,2.2) node {CHARIC};
  		\draw	[gray,<->] (3.8,.2) -- (3.8,3.8) ; \draw [gray] (3.8,2) node {CHARHT};
  		\draw	[gray,<->] (3.8,.2) -- (3.8,-1.3) ; \draw [gray] (3.8,-.5) node {CHARDP (=0 ici)};\draw (2.8,-1.3) -- (4.8,-1.3) ;
  		\draw [thick] (-.2,0.2) -- (10,0.2) ; 
  		\filldraw [white] (6.8,-1) -- (8.8,-1) -- (8.8,1) -- (6.8,1) -- cycle ; \draw [gray] (7.8,0) node {Indice};
  		\draw [red] (6.8,-1) -- (8.8,-1) -- (8.8,1) -- (6.8,1) -- cycle ; 
  		\filldraw [white] (8,3) -- (8,5) -- (10,5) -- (10,3) -- cycle ; \draw [gray] (9,4) node {Exposant}; 
  		\draw [red] (8,3) -- (8,5) -- (10,5) -- (10,3) -- cycle ;
  		\draw [red] (1.65,4.5) node {Texte};
  		\draw [red] (6.15,4.5) node {Math};
  		\draw [gray] (2.75,0.35) node {base line};
  	\end{tikzpicture}\end{center} 
	
	
	
	
	{  \LaTeX{} \textsc{ajoute des espaces en mode math}}
	 \begin{itemize}
		 \item<+->  Il y a 8 types de symboles, comme \texttt{mathord} et \texttt{mathbin} ;
		 \item<+->  entre 2 symboles,   \LaTeX{} ajoute éventuellement de l'espace.
		 \item<+->  Il y a trois espaces possibles :
		\begin{itemize}
			\item<+-> \verb=\thinmuskip= ou \verb=\,= qui vaut 3\,mu ;
			\item<+-> \verb=\medmuskip= ou \verb=\:= qui vaut 4\,mu plus 2\,mu minus 4\,mu ;
			\item<+-> \verb=\thickmuskip= ou \verb=\;= qui vaut 5\,mu plus 5\,mu ;
		\end{itemize}
		 \item<+->  Certains espaces entre symboles ne sont pas ajoutés quand on est en indice :
		 \[\textstyle\sum\limits_{i=1}^ni=\dfrac{n(n+1)}{2}\]
	 \end{itemize}
	
	
	{\textsc{On peut modifier ces espacements}}
	 \begin{itemize}
		 \item<+->  Pour modifier ces espaces, par exemple :
		\begin{itemize}
			\item<+-> \verb+\thinmuskip=2mu+
			\item<+-> \verb+\medmuskip=3mu minus 1mu+
			\item<+-> \verb+\thickmuskip=4mu plus 1mu minus 1mu+
		\end{itemize}
		 \item<+->  Ici, on a réduit l'espacement et l'élasticité.
	 \end{itemize}
	
	
	{\textsc{On peut modifier les tailles des indices et exposants}}
	 \begin{itemize}
		 \item<+->  Cela se fait de 2 façons :
		\begin{itemize}
			\item<+-> \verb+\DeclareMathSizes{1}{2}{3}{4}+ avec les tailles des polices :
\begin{enumerate}
	\item<+-> de texte, qui sert ici de base ;
	\item<+-> math principale, souvent égale à celle de texte ;
	\item<+-> math indice ou exposant, plus petite ;
	\item<+-> math indice ou exposant d'indice ou exposant ou \dots
	
						encore plus petite (ensuite, on garde la même taille) ;
\end{enumerate}
			\item<+-> On peut aussi déclarer :
\begin{itemize}
	\item<+-> \verb=\def\defaultscriptratio{.78}=
	\item<+-> \verb=\def\defaultscriptscriptratio{.62}=
	
						 par exemple.
	\item<+-> Cela donne des tailles d'indices pour toutes les tailles de caractères.
\end{itemize}
   \end{itemize}
   \end{itemize}
	
	
	
	{\textsc{Placer un indice en présence d'exposant}}
	 \begin{itemize}
		 \item<+->  Placer un indice \texttt{I} à un caractère \texttt{C} en présence d'un exposant \texttt{E} :
	 \begin{itemize}
		 \item<+->  depuis la ligne de base, descendre de \texttt{max(p+S19,S17)} 
		 						où \texttt{p} est la profondeur de \texttt{C}.
		 \item<+->  Calculer \texttt{h} le haut de \texttt{I} ;
				\begin{itemize}
					\item<+-> s'il le faut, descendre \texttt{I} de façon telle que $\mathtt{h}\leqslant4/5\times\mathtt{S5}$ ;
					\item<+-> s'il le faut, descendre \texttt{I} et monter \texttt{E} de façon telle que
										le haut de \texttt{I} et le bas de \texttt{E} soient distants d'au moins $4\times\mathtt{X8}$.
				\end{itemize}
		 \item<+->  Remarques :
				\begin{itemize}
					 \item<+->  \texttt{S} désigne les dimensions de \ogla \textsc{symbols} \fgla;
					 \item<+->  \texttt{X} désigne les dimensions de \ogla \textsc{largesymbols} \fgla;
					 \item<+->  \texttt{S5} est \texttt{xHeight}, la hauteur du \ogla x \fgla ;
					 \item<+->  \texttt{X8} est la hauteur de la barre de fraction.
				\end{itemize}
	 \end{itemize}
	 \end{itemize}
	
	\section{\textsc{Glyphes}}

\subsection{Type 1}

	
	{\textsc{Les courbes de Bézier sont :}}
	\begin{enumerate}
		\item<+->  de la forme : $\overrightarrow{OM}=\sum\limits_{k=0}^n\binom{n}{k}t^k(1-t)^{(n-k)}\overrightarrow{OA_k}$ ;
		\item<+->  $A_0$ et $A_n$ sont les extrémités ;
		\item<+->  les autres points sont les points de contrôle ;
		\item<+->  la tangente aux extrémités est dirigée vers le premier point de contrôle ;
		\item<+->  dans les polices, $n=3$, on a deux points de contrôle.
	\end{enumerate}
	
	
	
	{\textsc{Un exemple :}}
  	\begin{center}\begin{tikzpicture}
  		\filldraw [pink] (1,2) circle (2pt)
  		                 (0,1) circle (2pt);
  		\filldraw [green](0,0) circle (2pt)
  		                 (5,2) circle (2pt);
  		\draw [->] (0,0) .. controls (0,1) and (1,2) .. (5,2) ;
  		\draw	[gray,->] (0,0) -- (0,1) ;
  		\draw	[gray,->] (5,2) -- (1,2) ;
  	\end{tikzpicture}\end{center}
	On a symbolisé les extrémités en vert et les deux points de contrôle en rose
	
	On assemble des courbes de ce type pour obtenir un ou des chemins fermés.
	
	On remplit alors \textsl{à droite} des chemins fermés.
	
	
	{\textsc{Un exemple complet :}}
  \bigskip
  	\begin{center}\begin{tikzpicture}
  		\draw [->] (.59,3.36) .. controls (.88,4.11) and (1.50,4.58) .. (2.40,4.58) ;
  		\draw [red] (.59,3.36) -- (.88,4.11) -- (1.50,4.58) -- (2.40,4.58)
  					-- (2.71,4.58) -- (3.78,3.97) -- (3.78,2.98) --(3.78,.94) -- (3.78,.42) -- (4.05,.32) 
  					-- (4.47,.43) -- (4.51,.17) -- (4.51,.17) -- (3.72,.04) -- (3.25,-.07) 
  					-- (2.97,-.07) -- (3.00,.68) -- (3.00,.68) -- (2.13,-.12) -- (1.58,-.12) --
  					(.87,-.12) -- (.35,.33) -- (.35,1.07) -- (.35,1.88) -- (.97,2.12) -- (1.70,2.37)
  					-- (3.00,2.81) -- (3.00,3.58) -- (2.78,4.13) -- (1.91,4.10) -- (1.46,4.07) -- 
  					(1.17,3.60) -- (.94,3.16) ;
  		\draw [->] (2.40,4.58) .. controls (2.71,4.58) and (3.78,3.97) .. (3.78,2.98) ;
  		\draw [->] (3.78,2.98) -- (3.78,.94) ;
  		\draw [->] (3.78,.94) .. controls (3.78,.42) and (4.05,.32) .. (4.47,.43) ;
  		\draw [->] (4.47,.43) -- (4.51,.17) ;
  		\draw [->] (4.51,.17) .. controls (4.51,.17) and (3.72,.04) .. (3.25,-.07) ;
  		\draw [->] (3.25,-.07) .. controls (2.97,-.07) and (3.00,.68) .. (3.00,.68) ;
  		\draw [->] (3.00,.68) .. controls (2.13,-.12) .. (1.58,-.12) ;
  		\draw [->] (1.58,-.12) .. controls (.87,-.12) and (.35,.33) .. (.35,1.07) ;
  		\draw [->] (.35,1.07) .. controls (.35,1.88) and (.97,2.12) .. (1.70,2.37) ;
  		\draw [->] (1.70,2.37) -- (3.00,2.81) ;
  		\draw [->] (3.00,2.81) .. controls (3.00,3.58) and (2.78,4.13) .. (1.91,4.10) ;
  		\draw [->] (1.91,4.10) .. controls (1.46,4.07) and (1.17,3.60) .. (.94,3.16) ;
  		\draw [->] (.94,3.16) -- (.59,3.36) ;
  		\draw [red] (3.00,1.03) -- (2.77,.76) -- (2.28,.44) -- (1.91,.44) -- (1.48,.44) -- (1.20,.81) --
  		      (1.20,1.23) -- (1.20,1.95) -- (3.00,2.49) -- (3.00,2.49) ;
  		\draw [<-] (3.00,2.49) -- (3.00,1.03) ;
  		\draw [<-] (3.00,1.03) .. controls (2.77,.76) and (2.28,.44) .. (1.91,.44) ;
  		\draw [<-] (1.91,.44) .. controls (1.48,.44) and (1.20,.81) .. (1.20,1.23) ;
  		\draw [<-] (1.20,1.23) .. controls (1.20,1.95) and (3.00,2.49) .. (3.00,2.49) ;
  	\end{tikzpicture}\end{center}
	
	
	
	
	
		
		\[ e^{ \scriptscriptstyle\frac{A}{B}+ 1} \quad e^{\tfrac{\textscsl{a}}{\raisebox{0.3ex}{\textscsl{\scriptsize b}}} + 1} \quad e^{\tfrac{A}{B} + 1}\]%
		
		
		\[ % start display-style math mode 
e^{{\displaystyle\frac{A}{B}} + 1} % awful  
   \quad 
e^{\displaystyle\frac{A}{B}+1} 
   \quad
e^{\textstyle\frac{A}{B}+1} 
   \quad
e^{\frac{A}{B}+1} % "\scriptstyle" is default math mode
   \quad
e^{\scriptscriptstyle\frac{A}{B}+1} 
   \quad
e^{(\mkern-1.5muA/\mkern-2.5mu B+1)} % apply some (negative) kerning
   \quad
\exp\Bigl(\frac{A}{B}+1\Bigr) 
\]
		
		
		\begin{gather*}
 \mbox{\scriptsize%
    $A = \begin{bmatrix} a & b & c \\ d & e & f \\ g & h & i \end{bmatrix}$}\\
    A = \begin{bmatrix} a & b & c \\ d & e & f \\ g & h & i \end{bmatrix}
\end{gather*}
		
		
		
 

\tabulinestyle{on 2pt off 2pt}
$\begin{tabu}{cc|cc|cc|c}
  1 & -1 & 0 & & & & 0 \\
  -1 & 1 & 0 & 0 & 0 & & \\ \tabucline-
  0 & 0 & \bullet & \bullet & 0 & &\\
  & 0 & \bullet & \bullet & 0 & 0 & \\ \tabucline-
  &  & 0 & 0 & \bullet & \bullet & 0 \\
  &  &  & 0 & \bullet & \bullet & 0 \\ \tabucline-
  0 &  &  & 0 & 0 &0 & \ddots
\end{tabu}$

		

	
		totot
		lalal
		
		\[
		%% \let\nmatrix\bracketMatrixstack or \parenMatrixstack \braceMatrixstack, \vertMatrixstack, and simply \Matrixstack
		%% \fixTABwidth{<T or F>} 
		%% An optional argument exists to set the column alignment as l, c, or r
		%% 
	\nmatrix{1,2,3;4,5,6}
	\]
		
\[
\begin{matrix}
\framebox[3.\width]{J} &
\begin{matrix}
0 & 0  \\
0 & 0
\end{matrix}
& \\
\begin{matrix}
0 & 0 \\
0 & 0
\end{matrix}
& \framebox[3.\width]{J} &
\end{matrix}
\]
		    
\end{document}