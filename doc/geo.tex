%%% Local Variables: 
%%% coding: utf-8
%%% mode: latex
%%% TeX-engine: xetex
%%% End: 
\documentclass[12pt,a4paper,oneside]{article} % 10pt font size, A4 paper and two-sided margins

\usepackage[a4paper,text={16.5cm,25.2cm},centering]{geometry}
\usepackage{amsmath,amsthm,amsfonts,amssymb,mathtools}
\usepackage{color,graphicx,xcolor}
\usepackage{tikz}
\usetikzlibrary{decorations.pathreplacing}
\usepackage[prefix=rt]{xcolor-material}
% Output font encoding for international characters
%XeLaTeX is essentially a replacement for pdfLaTeX. It was primarily developed to enable better font handling. For latin writers, the main benefit of XeLaTeX is the ability to use the fonts on your computer, just as you can with other software.
\usepackage{fontspec,xltxtra,xunicode}
\defaultfontfeatures{}
%\usepackage{palatino} % Use the Palatino font
% to choose a font, you just need its full name on your system
% fc-list  | grep -i texgyr
% Here, the scales of the fonts have been chosento equalise their lowercase letter heights
\setmainfont{TeX Gyre Adventor}[Scale=MatchLowercase]
%\setmonofont{Inconsolata}[Scale=MatchLowercase]
%\usepackage{microtype} % Improves spacing
%\usepackage{fullpage}       % Smaller margins


\usepackage{fancyhdr} % For fancy headers
\pagestyle{fancy}



\newcommand{\entry}[4]{\markboth{#1}{#1}\textbf{#1}\ {(#2)}\ \textit{#3}\ $\bullet$\ {#4}}  % Defines the command to print each word on the page, \markboth{}{} prints the first word on the page in the top left header and the last word in the top right

%----------------------------------------------------------------------------------------



\newcommand{\concept}[1]{\textcolor{rtLightBlue900}{#1}} 
\newcommand{\subconcept}[1]{\textcolor{rtLightBlue900}{\textit{#1}}}
\usepackage{listings}


\lstdefinestyle{hama}{%
    literate={1}{\raisebox{0.5ex}{\fbox{\textcolor{blue}{1}}}}{1}%
        {0}{\fbox{\textcolor{red}{0}}}{1},%
    basicstyle=\ttfamily,%
}

\newcommand{\HAMA}[1]{\lstinline[style=hama]{#1}}

\newcommand{\Ra}[1]{\textcolor{rtLightBlue900}R\textsubscript{\textcolor{rtLightBlue900}{#1}}}
\newcommand{\Rab}[1]{\small{\textcolor{rtLightBlue900}{2}}\textcolor{rtLightBlue900}R\textsubscript{\textcolor{rtLightBlue900}{#1}}}
\newcommand{\gd}[1]{\textbf{δ}\textsubscript{\textbf{#1}}}
\newcommand{\gc}[1]{\textbf{γ}\textsubscript{\textbf{#1}}}
\newcommand{\gq}[1]{\textbf{ε}\textsubscript{\textbf{#1}}}
\newcommand{\Ca}[1]{\textcolor{rtLightBlue900}C\textsubscript{\textcolor{rtLightBlue900}{#1}}}



\begin{document}

Text\textsubscript{effect}
\Ra{1} \gd{1} \gc{1} \gq{1}
\section{Définitions des expressions mathematiques}

Quelques definitions pour clarifier le jargon.

%%\colorpalette[primary toggle at=600, secondary toggle at=A700, title text color=black]{rtLightBlue}


\entry{Axiome}{}{définition}{est une proposition évidente par elle-même.}


\entry{Problème}{}{définition}{est une question posée qui appele une solution.}

\entry{Hypothése}{}{définition}{est une suposition faite dans l'énoncé d'une proposition ou bien encore dans le courant d'une demonstration.}

\entry{Théorème}{}{définition}{est une vérité rendue evidente par un raisonement appelé \emph{demonstration}.}

\entry{Lemme}{}{définition}{est une vérité accesoire rendu neccessaire pour la demonstration d'un \emph{théorème} ou la solution d'un \emph{problème}.}

\entry{Proposition}{}{définition}{est un terme générique  pour théorème, problème ou lemme.}

\entry{Corollaire}{}{définition}{est la conséquence qui dérive d'une propositions.}

\entry{Scolie}{Scholie}{définition}{est une remarque sur des propositions tendant a mettre en lumiere leur liens ou leur utilité ainsi que leur restrictions ou extensions.}

Toute proposition consiste dans une hypothèse
et
une con-
clusion qui en découle, soit immédiatement, soit en vertu
d'un raisonnement qu'on appelle démonstration.
On nomme réciproque d'une proposition une seconde pro-
position dont l'hypothèse et la conclusion sont respectivement
la conclusion et l'hypothèse de la première. La proposition
contraire d'une proposition est une autre proposition dont
l'hypothèse et la conclusion sont respectivement la négation
de l'hypothèse et de la conclusion primitives. Ainsi, la propo-
« 5i G
sition « Si A égale B, C égale D » a pour réciproque
égale D, A égale B », et pour contraire « Si A n'est pas égal
à B, C n'est pas égal à D ».

La vérité de la réciproque d'une proposition exacte entraîne celle de la proposition contraire. Ainsi, soit
proposition «Si A égale B, C égale D »; de la réciproque : «Si C égale D, A égale B »


\section{La mesure de l'étendue}

On ne considère, en Mathématiques, que les grandeurs dont on peut définir
d'une manière précise l'égalité et l'addition;
Lorsqu'une grandeur est la somme de 2, 3, 4, ...
parties, égales à une autre grandeur de même espèce, on dit que la première est un multiple de la seconde et la seconde est une partie aliquote de la première.

Deux grandeurs sont dites commensurables entre elles lorsqu'elles
sont des multiples d'une troisième grandeur qu'on appelle alors leur commune mesure; dans le cas contraire, elles sont incommensurables entre elles.

Pour mesurer une grandeur, on cherche une commune mesure entre cette grandeur et
une autre de même espèce, arbitraire, mais bien connue, porte le nom d'unité.
mesurer une grandeur commensurables avec l'unité chercher combien cette grandeur renferme d' unités ou de parties  aliquotes de
l"unité
le nombre qui exprime sa mesure est entier n unites ou fractionnaire 1/n unite
 qu'il faut voir les conséquences de
De ces principes fondamentaux; les règles Arithmétique calcul des nombres
entiers ou fractionnaires.
\HAMA{00101101}

\section{Géometrie }

Un peu de lexique, le mot géométrie vient de  \entry{gê}{terre}{γn}{} et \entry{métron}{mesure}{μέτρον} (geômetrês) qui signifie « géomètre.la
science de l'étendue


Les mathématiques traitent la géometrie de bien des maniere; ici nous parlons de géometrie elementaire.

Pour notre propos, une definition simple est un bon point de départ:

\entry{Géometrie}{intuitive}{définition}{La connaissance des formes et des figures dans l'espace. L'etude des relations entre le point, la ligne et la figure dans un plan ou bien dans le volume de l'espace.Géométrie a pour objet l'étude des propriétés des
surfaces, de lignes
et
en particulier,
comme
son
nom
l'indique, la
figures,
mesure de
l'étendue.}

  la Géométrie plane,
aux figures situées dans un plan unique, et la Géo-
métrie dans l'espace, relative aux figures dont les éléments
peuvent être disposés d'une manière quelconque dans l'es-
divise la
:
relative
pace.
\entry{Forme}{}{définition}{Le volume d'un corps matériel
est l'étendue
du lieu que
ce corps occupe dans l'espace. Ce lieu est essentiellement
limité; sa limite, qui le sépare de l'espace environnant,
prend
nom
de surface. Les diverses faces d'un corps sont autant
de surfaces dont les limites ou les intersections mutuelles
s'appellent lignes. Enfin, on donne le nom de points aux li-
mites ou extrémités d'une ligne, aux intersections mutuelles
le
des lignes.est un terme generique pour n'importe quel figure ou ensemble de figures. }

\entry{Figure}{}{définition}{est un terme generique lui aussi qui designe tout aussi bien le carré que le cercle.On donne le nom de figure à un ensemble quelconque de surfaces, de lignes ou de points}

\entry{Point}{}{définition}{est un terme generique lui aussi qui designe tout aussi bien le carré que le cercle.}
La plus simple de toutes et les lignes est la ligne droite dont la notion est familière à tout un fil tendu offre l'image.
On nomme ligne brisée une ligne ABCD formée de plusieurs portions de droites placées bout à bout toutes les lignes autres que la ligne droite ou les lignes brizées se confond sous la dénomination commune de lignes courbes



La plus simple de toutes les surfaces est le plan, dont
un miroir peut donner l'idée. La définition géométrique
plan consiste en ce que toute droite qui joint deux points le cette surface y est contenue tout entière. C'est ainsi que,pour vérifier si table est plane, on s'assure qu'on peut y appliquer dans tous les sens une règle bien dressée, sans qu'il reste aucun vide entre la table et la règle.


\section{GÉOMÉTROGRAPHIE}
L'extension ne nécessite pas \LaTeX{} 

\entry{Géométrographie}{définition}{ʒeɔ.me.tʁɔ.ɡʁa.fi}{estArt des constructions géométriques .}

La théorie proprement dite qui n'est, en somme, que l'indication
de notations avec les conventions adoptées.

\subsection{NOTATIONS.}

Une notation géométrographique \emph{Bernes},  \gd{1} \gc{1} \gq{1}.

Une notation géométrographique \emph{Lemoine}, \Ra{1} \Ra{2} \Rab{1} \Ca{1} \Ca{2} \Ca{3}.


\Ra{1} \gd{1} \gc{1} \gq{1}

Symboles pour la règle : \Ra{1} \Ra{2} ou \gd{1} \gd{2}

Tracer une droite quelquonque : \Ra{2}  \gd{}

Faire passer le bord d'une règle par un point placé s'appellera l'\emph{opération} \Ra{1} ou \gd{1}, pour abréger, op.: (\Ra{1} \gd{1}); donc, spéculativement, faire bord d'une règle par deux points sera l'opération: (\Rab{1}  \gd{2}).


Tracer une ligne en suivant le bord de la règle sera (\Ra{2} ou \gd{2}).

Symboles pour le compas : \Ca{1} \gc{1} \Ca{2} \gc{2} \Ca{3}  \gc{}

Tracer un cercle quelquonque : (\Ca{3} \gc{}).


Mettre la pointe du compas en \emph{un point placé} sera op. (\Ca{1} ou \gc{1}); donc, spéculativement prendre avec le compas la distance de deux points placés sera op. (2\Ca{1} \gc{3}).

Tracer un cercle mais dont le centre est soit un point déterminé soit sur une ligne  : (\Ca{2} \gc{2}).




M. Bernes ne fait pas la distinction que j'établis entre (\Ca{1} \Ca{2}). Placer la pointe du compas en un point indéterminé d'une ligne tracée, c'est-a-dire ce que j'appelle \Ca{2}, il l'assimile a
\Ca{1}, c-a-d la pointe en un point déterminé. C'est d'ailleurs
une distinction dont 1 importance n'est que spéculative;


Symboles pour l'équerre :

Parallèle ou perpendiculaire quelconque à la ligne de terre au moyen de l'equerre ou du T : \gq{} .
 
 Ligne de rappel ou parallèle  à la ligne de terre passant par un point déterminé : \gq{1} .
 
Parallèle  quelconque à une droite :\gq{2} .
 
 Parallèle à une droite donnée passant par un point determiné  :\gq{3} .
 
 
 
 \subsection{coefficient de simplicité ou simplicité.}
 Nous supposerons que toute droite tracée et que tout cercle
tracé dans le cours d'une construction le sont en entier.

A la Géométrie canonique des Grecs, qui n'admet que les
solutions par la droite et le cercle, correspondra la Géornétrographie canonique qui admettra seulement la règle et le compas.


une construction ; en notation géométrographique \emph{Lemoine} ;s'exprimera par une formule : 

$[l_{1}.\Ra{1} + l_{2}.\Ra{2} + m_{1}.\Ca{1} + m_{2}.\Ca{2} + m_{3}.\Ca{3}]$.

Le nombre $l_{1} + l_{2}+ m_{1} + m_{2} + m_{3}$ est \emph{le coefficient de simplicité}.

Le nombre $l_{1} +  m_{1} + m_{2}$ est \emph{le coefficient d'exactitude}.

Le nombre $ l_{2}$ correspond au nombre de ligne tracées.

Le nombre $m_{3}$ correspond au nombre de cercles tracés.

une construction ; en notation géométrographique \emph{Bernes}  avec equerre,  \gd{1} \gc{1} \gq{1} ;s'exprimera par une formule : 

$[l.\gd{} + l_{1}.\gd{1} + l_{2}.\gd{2} + m.\gc{} + m_{1}.\gc{1} + m_{2}.\gc{2} + n.\gq{} + n_{1}.\gq{1} + n_{2}.\gq{2} + n_{3}.\gq{3} ]$ 


Le nombre $l + 2.l_{1} + 3.l_{2} + m + 2.m_{1} + 3.m_{2} + 4.m_{3} + n + 2.n_{1} + 3.n_{2} + 4.n_{3}$ est \emph{le coefficient de simplicité}.

Le nombre $l_{1} + 2.l_{2} + m_{1} + 2.m_{2} + 3.m_{3} + n_{1} + 2.n_{2} + 3.n_{3}$ est \emph{le coefficient d'exactitude}.

Le nombre $l + l_{1} + l_{2} + n + n_{1} + n_{2} + n_{3}$ correspond au nombre de ligne tracées.

Le nombre $ m + m_{1} + m_{2} + m_{3}$ correspond au nombre de cercles tracés.


a notation A(ρ) ou A(BC) désignera le cercle de centre A et de rayon ρ ou BC.


Je conviens de définir la simplicité d'une construction par son coeficient de simplicité; la construction géométrographique sera donc celle qui a le coeficient de simplicité le plus petit.




 L'application en discutant les constructions fondamentales classiques qui se trouvent partout
les mêmes, transmises séculairement par les géomètres depuis
les Grecs, dans tous les ouvrages de géométrie.


Je montre ainsi, dès le début, que ces constructions universellement enseignées peuvent, toutes à peu près, être notablement simplifiées, quelquefois dans des proportions qui semblent invraisembables, et que l'on est conduit à la notion d'un Art des constructions géométriques et à une methode pour les simplifier.

\section{}
Tracer une droite quelconque; op.

Tracer une droite qui passe par un point placé

Tracer une droite passant par deux points placés 
Tracer un cercle quelconque op. (C3).

Tracer un cercle quelconque dont le centre est placé;

VI. Prendre avec le compas une longueur donnée AB
op.
Tracer un cercle dont le rayon est une longueur donnée
et le centre un point place; op. 
VIII. Porter sur une ligne donnée, à partir d'un point indé-
terminé de cette ligne ou à partir d'un point placé
sur cette ligne, la longueur comprise entre les branches du compas

Tracer un angle droit ou tracer deux droites perpendicu-
laires entre elles.
a


\section{Huffman}

\concept{Huffman procedure} 
Procedure to design the optimal code.

\subconcept{1} Given prob $p_1,p_2,...,p_{k-1},p_k$. Start with the two smallest prob.

\subconcept{2} Group them together as the binary descendant of a node.

\subconcept{3} Repeat until one node is left.

\concept{Equivalence of PF codes and strategy for guessing via binary questions} TODO

\concept{Interpretation of entropy as expected number of questions for guessing the random variable} TODO





\concept{Fixed-to-Fixed Length Source Codes} 

Codes of type $U\to \{0,1\}^*$ or $U^n \to \{0,1\}^*$ are called \subconcept{fixed-to-variable} length codes, and all our designs have error free recovery of the source from its representation.


We want \subconcept{Fixed-to-fixed} codes $C:U^n \to \{0,1\}^k$ ($2^k$ representations), to obtain efficient codes we will give up error free recovery replace this by recovery with very small prob. of error.


The code assign binary representations only to a subset $S \subset U^n$ which ensure $Pr((u_1...u_n)\in S) \approx 1$ and $|S|\le 2^k$.



\section{Source Coding}

\concept{Introduction}
Diagram of a general communication system.
\subconcept{Discrete sources} output of the source is in discrete time and discrete valued.
\subconcept{Source Coding} representation of information sources in bits.
\subconcept{Source Code Function} $C:U\mapsto\{0,1\}^*=\{\emptyset,0,1,00,...\}$.

\concept{Non-Singular Codes} 
A code $C$ is \subconcept{singular} if $\exists u \neq v /\ C(u)=C(v)$. A code $C$ is \subconcept{non-singular} if it is not singular.
With a code $C$ define for a positive integer $n$ :
$C^n:U^n\mapsto\{0,1\}^*$ as $C^n(u_1,u_2,...,u_n)=C(u_1)C(u_2)...C(u_n)$
$C^*:U^*\mapsto\{0,1\}^*$ as $C^*(u_1u_2...u_n)=C(u_1)C(u_2)...C(u_n)$

\concept{Uniquely Decodable Codes}
A code $C$ is said to be \subconcept{uniquals decodable} if $C^*$ is non-singular.We want our codes to be uniquals decodable.

\concept{Prefix-Free Codes} 
A sequence $u_1,... u_n$ is a \subconcept{prefix} of $v_1,...,v_n$ if $n \geqslant m /\ u_1=v_1,...,u_m=v_m$.
A code $C$ is said to be \subconcept{prefix-free} if $\forall u \neq v$ $C(u)$ is not a prefix of $C(v)$.

\subconcept{Theorem} A prefix-free code is uniquely decodable.
(In a binary-tree representation of a PF code all codewords are found on the leaves).

\concept{Variable-to-Fixed Length Source Codes}
$\equiv$ Dual of Fixed-to-Variable length source coding
$\equiv$ Dictionary to sed source coding
\subconcept{Idea} Given an alphabet $U$, find a dictionary $D \subset U^*$, assign $\lceil log|D| \rceil$ bit binary representation to words in $D$, and then given $U_1U_2..$, parse it into $w_1,w_2...$ of dictionary words and represent each words by its binary description.

\concept{Sources} A source producer a sequence $u_1,u_2,u_3,...$ each $u_i \in U$ being random variables.
A \subconcept{memory-less} source is one where $u_1,u_2,...$ are independent.
A \subconcept{stationary} source is one where each $(u_i,...,u_{i+n-1})$ has the same statistics as $(u_1,...,u_n)$ for each $i$ and each $n$.
A memory-less and stationary source is equivalent to $u_1,u_2,...$ are \subconcept{independent, identically distributed (iid)}.

\concept{Expected codeword length}
$E\left[ length(C(u)) \right]$ average number of bits/letter the code uses to represent the source. We want to minimize it and $C$ to be uniquely decodable.

\section{Entropy}
\subconcept{Lemma} $ln(z)\le z^{-1}$ with eq if $z=1$.
\subconcept{Property} $0 \le H(U) \le log |U|$
\concept{Entropy as a lower-bound to the expected codeword length} 
\subconcept{Theorem} For any uniquely decodable code $C$ for a source $U$, we have $E\left[ length(C(u)) \right] \ge \sum_{u}p(u)log_2 \frac{1}{p(u)} \triangleq H(u)$

\concept{Existence of PF codes with average length at most entropy + 1}
\subconcept{Theorem} Given source $U$ there exists a PF code $C$ s.t. 
$E\left[ length(C(u)) \right] \ge H(u)+1$
 
\concept{Entropy of multiple random variables} 
\subconcept{Property} Suppose $U$ and $V$ are ind. RV.
Then $H(UV)=H(U)+H(V)$.
\subconcept{Observe} Suppose we have $U_1U_2...$ iid. If we use a code $C$ to represent $n$ letters at time., we will have $H(U_1...U_n) \le E\left[ length(C(U_1...U_n)) \right] \le H(U_1...U_n)+1$.
\subconcept{Also} $\frac{1}{n} H(U_1...U_n) = H(U_1)$ (iid of U).

\concept{Properties of optimal codes} 
\subconcept{1} If $p(u) < p(v)$ then $l(u) \ge l(v)$.
\subconcept{2} In an optimal PF code there are more than 2 longest codewords. If not the longest codeword can be shortered without violating the PF condition.
\subconcept{3} Among optimal codes, there is one for the two least probable symbols are siblings.


\end{document}
