#    C Versions of the splay-tree based compression program discussed in:
#
#		Applications of Splay Trees to Data Compression
#				     by
#			      Douglas W. Jones
#	     Communications of the ACM, Aug. 1988, pages 996-1007.
#
# If this text was E-mailed to you, save it in a file, then exit the mail
# mail program and use a text editor to delete any lines produced by
# the mailer from this file (all lines before these shell comments you are
# currently reading); there should be no trailing blanks on any lines of this
# text; any that appear should be removed.
#
# To install this software on a UNIX system:
#  1) create a directory (e.g. with the shell command mkdir splaystuff)
#  2) change to that directory (e.g. with the command cd splaystuff),
#  3) direct the remainder of this mail through sh (e.g. sh < ../savedmail).
# This will cause sh to create files in the new directory; it will do nothing
# else.  Then read README in the new directory for additional instructions.
#
cat > README <<\xxxxxxxxxx
  File: README
  Author: Douglas Jones
  Date: Oct. 6, 1994 (Strip out cryptographic features in 1989 release).
  Purpose: Description of the splay-tree based compression software

This software distribution includes the following files:

	README     (you are reading this)
	splay.1    A man page for the splay and unsplay programs; if these
			programs are installed for public use, this file
			should be put in /usr/contrib/man1/splay.1 or some
			such (depending on how your local site manages this).
	splay.c    Code for the splay compression utility; typically, this
			should be compiled with a command such as
			cc -O -o splay splay.c and then the executable
			file should be stored in /usr/contrib/bin or some such
			(depending on how your local site manages such things).
	unsplay.c  Code for the unsplay expansion utility;
			do for this what you did with splay.c
	splay.i    Code used by both splay.c and unsplay.c, needed to compile
			both of them.  Changes to the declarations of register
			variables in the splay procedure in this file may be
			needed to improve performance.  Some C preprocessors
			may have trouble with some of the macros used here; in
			this case, these should be changed to procedures.

You may want to put splay.c, unsplay.c, and splay.i in /usr/contrib/src/splay
or some such file (depending on how your local site manages such things).

The following changes were made in order to remove cryptographic features
from the previous release of this code:  In the UNIX man page, file splay.1,
all text pertaining to the cryptographic features has been commented out.
All deletions in the code have been marked with comments; all of these
comments include or are near comments including references to the deleted
variable called cryptflag, and all comments describe the nature of the
deleted code, in English.  The most important of the information disclosed
by these comments essentially duplicates the suggestions made in
the August 1988 paper in Communications of the ACM regarding cryptographic
applications of this algorithm.

  NOTICE:  Please read the copyright notices in the source files before
	selling them or putting them into production use.  Also note that
	we know of no patents, either in the United States or elsewhere,
	that cover the algorithms used in this code.

  NOTICE:  The development of this package was not funded by any agency.
	Jeffrey Chilton did the hard work as a student in the University of
	Iowa SSTP Summer Research Participation Program (SSTP = Student Science
	Training Program, a program that allows secondary school students to
	work in university research labs.  If this software is useful or even
	interesting to you, consider a donation to the University of Iowa
	Foundation, Iowa City, IA 52242, earmarked either for Computer Science
	or for Science Education (the former is where this work was done, the
	latter administers SSTP).

  NOTE:  This code was developed under BSD 4.3 UNIX.  Use under System V will
	require at least the following:  In splay.c and in unsplay.c, replace
	the include file <strings.h> with <string.h>.  In unsplay.c, line 106,
	replace the call to rindex with a call to strrchr.
xxxxxxxxxx
cat > splay.1 <<\xxxxxxxxxx
.\" File: splay.1
.\" Author: Jeffrey Chilton, PO Box 807, West Branch, IA 52358.
.\" Author: Douglas Jones, Dept. of Comp. Sci., U. of Iowa, Iowa City, IA 52242.
.\" Date: Oct. 6, 1994.
.\" Note: We assert no copyright on this man page.
.\" Language: nroff -man (also troff, eroff ...)
.\" Purpose: UNIX man page for Data compression program.
.\" Note: all text pertaining to cryptographic options has been commented out.
.\"
.TH SPLAY 1 "Oct. 6, 1994"
.UC 6
.SH NAME
splay, unsplay \- compress and uncompress data
.SH SYNOPSIS
.PU
.ll +8
.B splay
[
.B \-c
] [
.B \-f
] [
.B \-n
] [
.B \-v
] [
.B \-s
.I states
] [
.I filename
.\" ] [
.\" .B \-k
.\" .I keyfile
.\" ] [
.\" .B \-p
.\" .I password
]
.ll -8
.br
.B unsplay
[
.B \-c
] [
.B \-f
] [
.B \-n
] [
.I filename
.\" ] [
.\" .B \-k
.\" .I keyfile
.\" ] [
.\" .B \-p
.\" .I password
]
.SH DESCRIPTION
.I Splay
reduces the size of
.I filename
using prefix codes derived from a binary splay tree.  Whenever possible,
the file is replaced by one with the extension
.B "\&.S."
If no files are specified, the standard input is compressed to the
standard output.  Compressed files with names ending in
.B "\&.S"
can be restored to their original form using
.I unsplay.
.I Splay/unsplay
can also be used as a filter.
.PP
The
.B \-c
(``cat'') option makes
.I splay/unsplay
write to the standard output; no other files are changed or deleted.  If used,
.B \-c
must be specified before all other arguments except
.B \-f, \-n,
and
.B \-v.
.PP
The
.B \-f
option will force the compression or uncompression of the input file,
even if it does not actually shrink, or the corresponding output file
already exists.  If
.B \-f
is not given the user is prompted as to whether an existing
file should be overwritten and the splay command will fail if the
compressed file is larger than the original.  In case of failure,
the output file is deleted and the input file is left unchanged.
.B \-f
must be specified before
.I filename.
.PP
By default,
.I splay/unsplay
will delete the input file.  The
.B \-n
option suppresses this.
.PP
The
.B \-v
option causes the printing of the percentage reduction of the input file.
It also prints the number of bytes used to create the splay tree(s).
.PP
.I Splay
uses the splay prefix algorithm first presented in "Applications
of Splay Trees to Data Compression", D.W. Jones,
.I Communication of the ACM,
August 1988, pages 996-1007.  A balanced splay tree is first constructed by the
program.  As each character is pulled from the input text, it is assigned a
binary prefix code derived by following the path from the root to the proper
leaf.  At each compression, the tree is semi-splayed in such a way
that the distance from the root to the leaf is shortened.
In this way, the most frequently used characters acquire the shortest codes.
.PP
By default,
.I splay/unsplay
uses a Markov compression algorithm which associates a splay tree with each
of 32 states (unless encryption is requested, in which case the default is 1).
The number of states can be changed by the
.B \-s
option, followed by the desired number of states (1 - 256).
When compressing a character, the state used is determined
by the previous character divided by the total number of states.
For 64K machines, no more than 20 states are recommended; for text,
more than 32 states usually leads to little improvement, but about 100 states
is sufficient to compete with the UNIX
.I compress
command.  Digitized images can frequently be compressed better than
.I compress
command with only 1 state.
.PP
.\" .I Splay
.\" can be used for data encryption.
.\" By using the
.\" .B \-p
.\" option, the user may specify a password to be used as an encryption key.
.\" The
.\" .B \-k
.\" option may be specified to enable the program to use a file to hold the
.\" encryption key.  Before compressing or uncompressing the input, the initial
.\" tree used for data compression is semi-splayed about each byte of the key,
.\" thus making it very difficult to recover the compressed data without
.\" knowledge of the key.  Any number of
.\" .B \-p
.\" or
.\" .B \-k
.\" options may be used; if multiple keys are specified, they will be
.\" concatenated in the order specified.
.\" The
.\" .B \-p
.\" and
.\" .B \-k
.\" options must be specified after the filename, if any is given.
.\" .PP
.\" Any time an encryption key is specified,
.\" .I splay
.\" prefixes several random bytes to the source file in order to complicate
.\" known-text attacks.  These bytes are automatically discarded by
.\" .I unsplay.
.\" If a key is specified,
.\" .I splay
.\" will retain the output even if it is larger than the input.
.\" .PP
.SH "DIAGNOSTICS"
Error: bad
.B \-x
argument
Usage:
.I splay/unsplay
[\-c] [\-f] [\-n] [\-v] [\-s states] [filename] [\-k keyfile] [\-p password]
.in +8
Invalid options were specified on the command line.
-c must precede all other arguments.
.\" Keys must be specified after filename.
.\" -s must be specified before keys.
.in -8
Error:
.IR splay/unsplay
can't open
.IR filename
.in +8
Unable to open this file.  File does not exist, or is inaccessible.
.in -8
Error:
.IR filename
is not in splayed format
.in +8
.I Unsplay
cannot unsplay the file, because it has not been splayed.
All splayed files are given the magic number ^S^P.
.in -8
Error: corrupt input in
.IR filename
.in +8
.\" Either the file has been encrypted and the proper decoding key has
.\" not been entered, or
the splayed data has been changed or ruined.
.in -8
Error: cannot
.IR splay/unsplay
.IR filename: .S
suffix expected
.in +8
Input file is required to have a .S suffix for unsplay to be able to use it.
.in -8
Error:
.IR filename
could not be compressed
.in +8
.I Splay
was unable to produce a smaller file.
.in -8
Error: too many arguments
.in +8
More than one filename has been specified or number of states specified is in
conflict with a previous setting.
.in -8
Error: \-s
.B x
not in bounds of 1 to 256
.in +8
The number of states is not within the range 1 - 256.
.in -8
.SH "SEE ALSO"
.br
crypt(1), compress(1), uncompress(1), zcat(1)
.SH BUGS
.\" As with most encryption algorithms, we cannot ensure the security of the
.\" encrypted result.  As of this writing (Oct 5, 1994), we know nothing
.\" about deciphering encrypted text produced using splay trees and export of
.\" cryptographic software utilizing this algorithm is limited by federal
.\" export controls.  In the short run, this may provide a small degree of
.\" security, but in the long run, it should only reinforce the insecurity of
.\" using this method.
.PP
For all of the text we have compressed and uncompressed, the output of the
.I unsplay
program has been the same as the original input to the
.I splay
program, but we cannot warrant that this will be the case for files
we have not tried.  In over five years of use, we have heard of no problems.
.PP
Unlike
.I compress
and
.I uncompress,
this package makes no effort to preserve the ownership modes, access times,
and modification times of the files processed.
This package will compress or uncompress only one file at a time.
.SH AUTHORS
Douglas W. Jones and Jeffrey E. Chilton
xxxxxxxxxx
cat > splay.c <<\xxxxxxxxxx
/* File: splay.c
   Author: Jeffrey Chilton, PO Box 807, West Branch, IA 52358.
   Author: Douglas Jones, Dept. of Comp. Sci., U. of Iowa, Iowa City, IA 52242.
   Date: Oct. 6, 1994
	 (revision of Feb 14, 1990 to delete cryptographic options).
	 (minor revision of Feb. 20, 1989 to add exit(0) at end of program).
	 (minor revision of Nov. 14, 1988 to fix portability problems).
	 (minor revision of Aug. 8, 1988 to eliminate unused vars, fix -c).
   Copyright 1988 by Jeffrey Chilton and Douglas Jones.
	      Copies of this program and associated files may not be sold,
			nor may it (or parts of it) be incorporated into
			products which will be sold, without the express
			prior permission of the authors.
	      Permission is hereby granted to make copies of this program for
			research and personal use so long as this copyright
			notice is included in the copy and in any derived work.
   Language: C (UNIX)
   Purpose: Data compression program
   Algorithm: Uses a splay-tree based prefix code, with one splay tree per
			state in a Markov model.  The nature of the Markov
			model is determined by the splay procedure in the
			include file splay.i.
*/

#include <stdio.h>
#include <strings.h>
#include <time.h>
#include "splay.i"

/* begin TRANSMIT macro */
#define MAXBITCOUNTER 8
short int bitbuffer = 0;
short int bitcounter = 0;
#define TRANSMIT(b) {						\
	bitbuffer = (bitbuffer << 1) + (b);			\
	if ((++bitcounter) == MAXBITCOUNTER) {			\
		putc(bitbuffer, out);				\
		bitcounter = bitbuffer = 0;			\
	}							\
}
/* end TRANSMIT macro */

short int stack[SUCCMAX]; /* used by compress, statically allocated for speed */

/* begin COMPRESS macro */
/* COMPRESS macro is used where speed is essential, and if your compiler */
/* can't handle the macro, or if you are out of memory, use the function */
/* compress and delete the macro */
#define COMPRESS(plain) {					\
	short int *U, *R;					\
	short int sp = 0;					\
	short int a = (plain) + SUCCMAX;			\
	R = &right[state * SUCCMAX];				\
	U = &up[state * SUCCTWICE];				\
	do { 		/* walk up the tree pushing bits */	\
		stack[sp] = R[U[a]] == a;			\
		++sp;						\
		a = U[a];					\
 	} while (a != ROOT);					\
	do {							\
		TRANSMIT(stack[(--sp)]);			\
	} while (sp != 0);					\
	SPLAY((short int)(plain));				\
}
/* end COMPRESS macro */

/* begin compress function */
compress(plain)
short int plain;
{
	COMPRESS(plain);
}
/* end compress function */

/* begin openfiles function */
openfiles()
{
	if (filename == NULL) {
		in = stdin;
		out = stdout;
		cflag = 1;
		compflag = 1;
		rmfileflag = 0;
	} else {
		if ((in = fopen(filename, "r")) == NULL) cannot_open(filename);
		if (cflag == 0) {
			strncpy(filenmod, filename, 117);
			strncat(filenmod, ".S", 3);
			if ((out = fopen(filenmod, "a")) == NULL) {
				cannot_open(filenmod);
			}
			if ((forceflag == 0) && (ftell(out) != 0)) overwrite();
		} else out = stdout;
	}
}
/* end openfiles function */

/* THE MAIN PROGRAM */
main(argc, argv)
int argc;
char *argv[];
{
	int plain;
	long int oldlen, newlen;

	linearg(argc, argv);

	if (states == 0) {
		states = nokeydefault;
		initsplay();
	}

	putc(MAGIC1, out);
	putc(MAGIC2, out);
	if (states == 256) {
		putc(NULL, out);
	} else {
		putc((char)states, out);
	}

	/* begin transmit garbage prefix */

	/* this section of code has been deleted.  This code section was
           conditional on the flag cryptflag.  If the flag was set, a
	   random number was made by multiplying the least significant
	   fifteen bits of the process ID (as returned by getpid()) times
	   the least significant fifteen bits of the time of day (as
	   returned by the time(NULL) function).  The 32 bit product was then
	   output, least significant byte first, using compress() to transmit
	   each byte.  This random prefix on the compressed data stream
	   should help complicate known text attacks on the encrypted
	   code */
	
	/* end transmit garbage prefix */

	/* begin compress input */
	for (;;) {
		if (((plain = getc(in)) == EOF) && feof(in)) break;
		COMPRESS ((short int)(plain & 0xff));
	}
	compress(MAXCHAR);
	/* end compress input */

	/* begin flushtransmit routine */
	while (bitcounter != 0) {
		TRANSMIT(0);
	}
	/* end flushtransmit routine */

	/* begin calculate compression */
	oldlen = ftell(in);
	newlen = ftell(out);
	if (vflag == 1) {
		int bytes = 2 * (states * SUCCMAX * sizeof(short)) + (states *
			SUCCTWICE * sizeof(short));
		float comp = 100.0 * (1.0 - ((float)newlen / (float)oldlen));
		fprintf(stderr, "%s: Compression: %2.1f%% ", filename, comp);
		if (rmfileflag == 1) {
			fprintf(stderr, "-- replaced with %s\n", filenmod);
		} else fprintf(stderr, "\n");
		fprintf(stderr, "Data structure used %d bytes\n", bytes);
	}
	/* end calculate compression */

	if ((newlen > oldlen) && (compflag == 0)) {
		unlink(filenmod);
		already_compressed();
	}
	if ((filename != NULL) && (rmfileflag == 1)) unlink(filename);
	exit(0);
}
xxxxxxxxxx
cat > unsplay.c <<\xxxxxxxxxx
/* File: unsplay.c
   Author: Jeffrey Chilton, PO Box 807, West Branch, IA 52358.
   Author: Douglas Jones, Dept. of Comp. Sci., U. of Iowa, Iowa City, IA 52242.
   Date: Oct. 6, 1994
	 (revision of Feb 14, 1990 to delete cryptographic options).
	 (minor revision of Feb. 20, 1989 to add exit(0) at end of program).
	 (minor revision of Nov. 14, 1988 to detect corrupt input better).
	 (minor revision of Aug. 8, 1988 to eliminate unused vars, fix -c).
   Copyright 1988 by Jeffrey Chilton and Douglas Jones.
	      Copies of this program and associated files may not be sold,
			nor may it (or parts of it) be incorporated into
			products which will be sold, without the express
			prior permission of the authors.
	      Permission is hereby granted to make copies of this program for
			research and personal use so long as this copyright
			notice is included in the copy and in any derived work.
   Language: C (UNIX)
   Purpose: Data uncompression program
   Algorithm: Uses a splay-tree based prefix code, with one splay tree per
			state in a Markov model.  The nature of the Markov
			model is determined by the splay procedure in the
			include file splay.i.
*/

#include<stdio.h>
#include<strings.h>
#include"splay.i"

/* begin RECEIVE macro */
#define TOPBITINBUFFER 128
#define MAXBITCOUNTER 8
short int bitbuffer = 0;
short int bitcounter = 0;
#define RECEIVE(bit) {				\
	short int  ch;				\
	if(bitcounter == 0) {			\
		ch = getc(in);			\
		if ((ch == EOF) && feof(in)) {	\
			bad_data();		\
		}				\
		bitbuffer = ch;			\
		bitcounter = MAXBITCOUNTER;	\
	}					\
	--bitcounter;				\
	if ((bitbuffer & TOPBITINBUFFER) != 0) {\
		bit = 1;			\
	} else {				\
		bit = 0;			\
	}					\
	bitbuffer = bitbuffer << 1;		\
}
/* end RECEIVE macro */

short int plain;	/* most recent character uncompressed */

/* begin UNCOMPRESS macro */
/* if your compiler cannot handle macros or you are out of memory, */
/* use the function only and delete the macro */
#define UNCOMPRESS()				\
{						\
	short int *R, *L, bit;			\
	short int a = ROOT;			\
	L = &left[state * SUCCMAX];		\
	R = &right[state * SUCCMAX];		\
						\
	do {  /* once for each bit on path */	\
		RECEIVE(bit);			\
		if (bit == 0) {			\
			a = L[a];		\
		} else {			\
			a = R[a];		\
		}				\
	} while (a <= MAXCHAR);			\
	plain = a - SUCCMAX;			\
	SPLAY(plain);			\
}
/* end uncompress macro */

/* begin uncompress function */
uncompress()
{
	UNCOMPRESS();
}
/* end uncompress function */

/* begin openfiles function */
openfiles()
{
	int s;

	if(filename == NULL) {
		in = stdin;
		out = stdout;
		cflag = 1;
		compflag = 1;
		rmfileflag = 0;
	} else {
		if ((in = fopen(filename, "r")) == NULL) {
			cannot_open(filename);
		}
		if (cflag == 0) {
			char *c;
			strncpy(filenmod, filename, 120);
			c = rindex(filenmod, '.');
			if ((c == NULL) || (c[1] != 'S') || (c[2] != '\000')) {
				no_S();
			}
			c[0] = c[1] = '\000';
			if ((out = fopen(filenmod, "a")) == NULL) {
				cannot_open(filenmod);
			}
			if ((forceflag == 0) && (ftell(out) != 0)) {
				overwrite();
			}
		} else {
			out = stdout;
		}
	}
	if ((getc(in) != MAGIC1) || (getc(in) != MAGIC2)) {
		not_splayed();
	}
	s = getc(in);
	if (s == NULL) {
		s = 256;
	}
	if ((states != s) && (states != 0)) {
		toomanyargs();
	}
	states = s;
	initsplay();
}
/* end openfiles function */

/* THE MAIN PROGRAM */
main(argc, argv)
int argc;
char *argv[];
{
	linearg(argc, argv);

	/* begin remove garbage prefix on encrypted data */

	/* this section of code has been deleted.  This code section was
           conditional on the flag cryptflag.  If the flag was set, the
	   first few bits of the compressed file are nonsense that should
	   be removed by four successive calls to the uncompress() function.
	   The four bytes of data returned by uncompress() can be discarded.
	   The random prefix discarded here should help complicate known
	   text attacks on the encrypted code */
	
	/* end remove garbage prefix on encrypted data */

	/* begin uncompress input */
	for (;;) {
		UNCOMPRESS();
		if (plain != MAXCHAR) {
			putc(plain, out);
		} else {
			break;
		}
	}
	/* end uncompress input */

	/* see if file really ended */
	if ((getc(in) != EOF) || !feof(in)) {
		bad_data();
	}

	/* finish the job */
	if (in == stdin) {
		exit(0);
	}
	if (rmfileflag == 1) {
		unlink(filename);
	}
	exit(0);
}
xxxxxxxxxx
cat > splay.i <<\xxxxxxxxxx
/* File: splay.i
   Author: Jeffrey Chilton, PO Box 807, West Branch, IA 52358.
   Author: Douglas Jones, Dept. of Comp. Sci., U. of Iowa, Iowa City, IA 52242.
   Date: Oct. 6, 1994
	 (revision of Dec 11, 1989 to delete cryptographic options).
         (very minor revision of Feb. 20, 1989 to eliminate %i in sscanf).
	 (minor revision of Nov. 14, 1988 to remove dead code).
	 (minor revision of Aug. 8, 1988 to correct declaration of filenmod).
   Copyright 1988 by Jeffrey Chilton and Douglas Jones.
	      Copies of this program and associated files may not be sold,
			nor may it (or parts of it) be incorporated into
			products which will be sold, without the express
			prior permission of the authors.
	      Permission is hereby granted to make copies of this program for
			research and personal use so long as this copyright
			notice is included in the copy and in any derived work.
   Language: C (UNIX)
   Purpose: Include file for splay-tree based compression.
   Algorithm: Procedures for command-line parsing and splay-tree maintenance.
*/

#define MAGIC1 0x93	/* ^S with the high bit set */
#define MAGIC2 0x10	/* ^P */
#define MAXCHAR	256	/* maximum source character code */
#define SUCCMAX	257	/* MAXCHAR + 1 */
#define TWICEMAX 513	/* 2 * MAXCHAR + 1 */
#define SUCCTWICE 514	/* TWICEMAX + 1 */
#define ROOT	1
#define nokeydefault 32	/* default number of states for no keys */
#define keydefault 1	/* default number of states for key */
short int *left;
short int *right;
short int *up;

/* begin SPLAY macro called from loops */
#define SPLAY(plain) {							\
	register short int *L, *R, *U, a, b, c, d;			\
	L = &left[state * SUCCMAX];					\
	R = &right[state * SUCCMAX];					\
	U = &up[state * SUCCTWICE];					\
	a = (plain) + SUCCMAX;						\
									\
	do { 	/* walk up the tree semi-rotating pairs of nodes */	\
		if ((c = U[a]) != ROOT) { 	/* a pair remains */	\
			d = U[c];					\
			b = L[d];					\
			if (c == b) {					\
				b = R[d];				\
				R[d] = a;				\
			} else {					\
				L[d] = a;				\
			}						\
			if (L[c] == a) {				\
				L[c] = b;				\
			} else { 					\
				R[c] = b;				\
			}						\
			U[a] = d;					\
			U[b] = c;					\
			a = d;						\
		} else { 						\
			a = c;						\
		}							\
	} while (a != ROOT);						\
	state = (plain) % states;					\
}
/* end SPLAY macro */

/* the following flags do not correspond directly to the command line args */
/* the documented meanings correspond to the default initial values */
int compflag = 0;	/* if output bigger delete it; set by -c, -f, -p, -k */
int rmfileflag = 1;	/* remove input when done; unset by -n, -c */
int vflag = 0;		/* don't show statistics; set by -v */
int forceflag = 0;	/* don't ignore overwrite; set by -f */
int cflag = 0;		/* output to stdout; set by -c */

/* an integer variable called cryptflag, with a default value zero, was
   declared here.  This was set by the -p or -k options */

/* the following variables determine the operation of the Markov model */
int states = 0;		/* number of states = 0; default indicates unknown */
int state = 0;		/* current splay tree; initial state is always 0 */

/* files */
char *filename = NULL;	/* textual name of the input file; NULL = unknown */
char filenmod[120];	/* textual name of the output file */
char *pgmname;		/* name of program for error messages */
FILE *in, *out;		/* files used for compression or uncompression */

/* begin line argument check function */
linearg(argc, argv)
int argc;
char *argv[];
{
	int i;
	pgmname = argv[0];

	for (i =1; i < argc; i++) {
		if (argv[i][0] == '-') { /* a flag */
                        if (argv[i][1] == 'p') {
                                char *p;
                                if (argv[i][2] != NULL) {
                                        p = &argv[i][2];
                                } else if ((i + 1) < argc) {
                                        p = &argv[i + 1][0];
                                        i++;
                                } else {
                                        syntax("-p");
				}
                                if (in == NULL) {
                                        openfiles();
                                }
                                if (states == 0) {
                                        states = keydefault;
                                        initsplay();
                                }
                                while (*p != NULL) {
					p++;
					/* the above nonsense code replaces
					   the code to use the successive
					   letters of the text following
					   the -p option as an encryption
					   key.  This was done by passing
					   each letter pointed to by p, as
					   a parameter, to the splay
					   function */
                                }
				/* the code that was here set the flags
				   compflag and cryptflag to signal that
				   an encryption key had been provided */
                        } else if (argv[i][1] == 'k') {
                                char *p;
                                FILE *f;
                                char c;
                                if (argv[i][2] != 0) {
                                        p = &argv[i][2];
                                } else if ((i + 1) < argc) {
                                        p = &argv[i + 1][0];
                                        i++;
                                } else {
                                        syntax("-k");
                                }
                                f = fopen(p, "r");
                                if (f == NULL) {
                                        cannot_open(p);
                                }
                                if (in == NULL) {
                                        openfiles();
                                }
                                if (states == 0) {
                                        states = keydefault;
                                        initsplay();
                                }
                                for (;;) {
                                        if (((c = getc(f)) == EOF)
                                                        && feof(f)) break;
					/* as given here, this does nothing
					   useful; in the original,
					   successive letters of the key
					   file were used as an encryption
					   key by passing each letter, as
					   a parameter, to the splay
					   function */
                                }
				/* the code that was here set the flags
				   compflag and cryptflag to signal that
				   an encryption key had been provided */
                        } else if (argv[i][1] == 'n') {
				rmfileflag = 0;
			} else if (argv[i][1] == 'v') {
				vflag = 1;
			} else if (argv[i][1] == 'f') {
				forceflag = 1;
				compflag = 1;
			} else if (argv[i][1] == 'c') {
				if (in != NULL) {
					syntax("-c");
				}
				cflag = 1;
				compflag = 1;
				rmfileflag = 0;
			} else if (argv[i][1] == 's') {
				char *p;
				int s;
				if (argv[i][2] != NULL) {
					p = &argv[i][2];
				} else if ((i + 1) < argc) {
					p = &argv[++i][0];
				} else syntax(argv[i]);
				if (sscanf(p, "%d\0", &s)
						== EOF) {
					syntax(argv[i]);
				}
				if (s < 1 || s > 256) {
					out_of_bounds(s);
				}
				if ((states != s) && (states != 0)) {
					toomanyargs();
				}
				if (states == 0) {
					states = s;
					initsplay();
				}
			} else {
				syntax(argv[i]);
			}
		} else { /* no - at front of argument */
			filename = &argv[i][0];
			if (in != NULL) {
				toomanyargs();
			}
			openfiles();
		} /* end if */
	} /* end for loop */
	if (in == NULL) {
		openfiles();
	} /* end if */
} /* end linearg */

/* begin build initial splay tree */
initsplay ()
{
	short int *L;
	short int *R;
	short int *U;

	left = (short *) malloc(states * SUCCMAX * sizeof(short));
	right = (short *) malloc(states * SUCCMAX * sizeof(short));
	up = (short *) malloc(states * SUCCTWICE * sizeof(short));
	for (state = 0; state < states; ++state) {
		short int i, j;
		L = &left[state * SUCCMAX];
		R = &right[state * SUCCMAX];
		U = &up[state * SUCCTWICE];
		for (i = 2; i <= TWICEMAX; ++i)	{
			U[i] = i/2;
		}
		for (j = 1; j <= MAXCHAR; ++j)	{
			L[j] = 2 * j;
			R[j] = 2 * j + 1;
		}
	}
	state = 0;
}
/* end build initial splay tree */

/* begin splay function */
splay(plain)
short int plain;
{
	SPLAY(plain);
}
/* end splay function */

/* begin overwrite permission */
overwrite()
{
	fprintf(stderr, "%s already exists;", filenmod);
	fprintf(stderr, " do you wish to overwrite (y or n)? ");
	for (;;) {
		char c[20];
		int i;
		if (fgets(c, 20, stdin) == NULL) {
			exit(-1);
		}
		i = strspn(c, " ");
		if ((c[i] == 'n') && (c[i + 1] == '\n')) {
			exit(0);
		} else if ((c[i] == 'y') && (c[i + 1] == '\n')) {
			freopen(filenmod, "w", out);
			break;
		} else {
			fprintf(stderr, "Please enter y or n! ");
		}
	}
}
/* end overwrite permission */

/* begin error messages */
cannot_open(name)
char *name;
{
	fprintf(stderr, "Error: %s can't open %s\n", pgmname, name);
	exit(-1);
}

not_splayed()
{
	fprintf(stderr, "Error: %s is not in splayed format\n", filename);
	exit(-1);
}

syntax(c)
char *c;
{
	fprintf(stderr, "Error: bad %s argument\nUsage: %s [input file] ",
		c, pgmname);
	fprintf(stderr, "[-s states] [-c] [-f] [-n] [-v] [-k keyfile] [-p ");
	fprintf(stderr, "password]\n");
	exit(-1);
}

bad_data()
{
	fprintf(stderr, "Error: corrupt input in %s\n", filename);
	exit(-1);
}

no_S()
{
	fprintf(stderr, "Error: cannot %s %s: .S suffix expected\n", pgmname,
		filename);
	exit(-1);
}

already_compressed()
{
	fprintf(stderr, "Error: %s could not be compressed\n",filename);
	exit(-1);
}

toomanyargs()
{
	fprintf(stderr, "Error: too many arguments\n");
	exit(-1);
}

out_of_bounds(s)
int s;
{
	fprintf(stderr, "Error: -s %d not in bounds of 1 to 256\n", s);
	exit(-1);
}
/* end error messages */
xxxxxxxxxx
